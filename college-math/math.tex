\documentclass[12pt]{article}
\usepackage{xeCJK}
\usepackage[utf8]{inputenc}
\usepackage{amsmath, amssymb, amsthm, mathtools, subfiles}

% Tree diagram
\usepackage{tikz}
\usetikzlibrary{trees}
% \usepackage{verbat}

% Drawing
\usepackage{pgf,pgfplots}
\pgfplotsset{compat=1.15}
\usepackage{mathrsfs}
\usetikzlibrary{arrows}
\pagestyle{empty}
\definecolor{rvwvcq}{rgb}{0.08235294117647059,0.396078431372549,0.7529411764705882}

% Hyperlink ref
% Be careful, it has to be the last package to be imported
\usepackage{hyperref}
% PDF-specific options
\hypersetup{
    colorlinks=true,
    linkcolor=blue,
    filecolor=magenta,
    urlcolor=cyan,
    pdftitle={Math},
    bookmarks=true,
}

% simple way to define
\usepackage{geometry}
\geometry{
  a4paper,
  total={170mm,257mm},
  left=20mm,
  top=20mm,
}


\setlength{\parindent}{0em} % paragragh indent
\setlength{\parskip}{0.6em}  % or use \bigskip \medskip \smallskip in content
\linespread{1.25} % line spacing

\title{MATH}
% \date{2019}
% \author{zhb}

\begin{document}
% TOC(table of contents) depth
\setcounter{tocdepth}{2}
\tableofcontents

\pagenumbering{gobble}
\newpage
\pagenumbering{arabic}

\section{MISC}

NZQRC: NaZi QRC(quick response code), $\mathbb{N} \subseteq \mathbb{Z} \subseteq \mathbb{Q} \subseteq \mathbb{R} \subseteq \mathbb{C}$\\
N:nature  Z:(From German Zahlen)  Q:rational number R:real number  C:complex\\
N and Z 是可数的无限集合\\
$\mathbb{N}$ is $\{0,1,2,\ldots\}$\\
$\mathbb{Z}$ is $\{\ldots,-3,-2,-1,0,1,2,3,\ldots\}$\\
$\mathbb{Q}$=整数+分数,可以表达为两个整数比的数,小数不分有限或者循环\\
$\mathbb{R}$=有理数+无理数

perpendicular 垂直\\
orthogonal 正交\\
dot plot 点图\\
symmetrical 对称\\
right-tailed 右侧\\
skew 歪斜,偏的\\
outlier 异常值,离群值\\
mean 平均数\\
median 中位数,排序后的中间值\\
mode 众数\\
prod(product) \\
progression 级数 数列\\
arithmetic progression 等差数列\\
geometric progression 等比数列\\
harmonic(和声、音乐般的、谐波) progression 调和级数\\
mathematical induction 数学归纳法\\
proper fraction 真分数\\
improper fraction 假分数

充分和必要,$\rightarrow Q \rightarrow$, Cupid's arrow(丘比特之箭),充分必要,前后,条件能推导Q,那就是Q的充分条件,如果Q能推导出另一个,那么这个条件就是Q的必要条件,前面如果 P->Q->P会易混

可导(可微) 连续 可积 有界:一个导弹连集结,$\rightarrow$

$tan'x=sec^2x$, 谈话2seconds

Some basic algebraric formula:\\
$(a+b)^2=a^2+2ab+b^2$ let $b=-b$ we have $(a-b)$\\
$(a+b)^3=a^3+b^3+3ab(a+b)$ let $b=-b$ \\
$(a+b+c)^2=a^2+b^2+c^2+2ab+2bc+2ac$\\
$(a+b+c)^3=a^3+b^3+c^3+3a^2b+3a^2c+3b^2c+3b^2a+3c^2a+3c^2b+6abc$\\
if $a+b+c=0$ then $a^3+b^3+c^3=3abc$\\
$a^3-b^3=(a-b)(a^2+ab+b^2)$ let $b=-z$ $\Rightarrow a^3+z^3$\\
$x^4+4y^4=(x^2+2y^2+2xy)(x^2+2y^2-2xy)$ //Sophie Germain's identity\\

和角公式的推导,一般从$\cos(\alpha-\beta)=\cos\alpha\cos\beta+\sin\alpha\sin\beta$开始推导,hint\footnote{有几种方法:1. 单位圆,余弦定理 2.向量 3.画图}

$\sin A + \sin B = 2\sin(\frac{A+B}{2})\cos(\frac{A-B}{2})$\\

e(exponential's first letter),无理数,自然对数的底数,sometimes called “Euler's number”, $=2.7182818284590\cdots$\\
$e=\lim_{n\to \infty}(1+\frac{1}{n})^{n}=\sum_{n=0}^{\infty}\frac{1}{n!}=\frac{1}{0!}+\frac{1}{1!}+\frac{1}{2!}+\cdots$

\newpage
\section{直线和平面}
\subsection{line直线}
In $\mathbb{R}^3$过$P_1(x_1,y_1,z_1), P_2(x_2,y_2,z_2)$的直线\\
$
L=\{\vec p_1 + t(\vec p_2 - \vec p_1) | t \in \}
\Rightarrow
\begin{bmatrix}
x \\ y \\z
\end{bmatrix}
=
\begin{bmatrix}
x_1 \\ y_1 \\ z_1
\end{bmatrix}
+t
\begin{bmatrix}
x_1-x_2 \\ y_1-y_2 \\ z_1-z_2
\end{bmatrix}
\Rightarrow
\left \{
  \begin{array}{l}
    x=x_1+t(x_1-x_2)\\
    y=y_1+t(y_1-y_2)\\
    z=z_1+t(z_1-z_2)\\
  \end{array}
\right.
$\\
即是参数式方程

\subsubsection{直线($\mathbb{R}^3$)}
key: 方向向量(l,m,n)\\
1. 一般式:
$
\left \{
    \begin{array}{l}
      A_1x+B_1y+C_1z=0\\
      A_2x+B_2y+C_2z=0
    \end{array}
  \right.
 $
从此也可以推出s=
$
\begin{vmatrix}
i & j & k\\
A_1&B_1&C_1\\
A_2&B_2&C_2
\end{vmatrix}
$

2. 标准式
$\dfrac{x-x_{0}}{l}=\dfrac{y-y_{0}}{m}=\dfrac{z-z_{0}}{n}$

3. 两点式,两点求出方向向量,转化为标准式

4. 参数式,
$\dfrac{x-x_{0}}{l}=\dfrac{y-y_{0}}{m}=\dfrac{z-z_{0}}{n}=t
\Rightarrow
\left \{
    \begin{array}{l}
      x=x_0+lt\\
      y=y_0+mt\\
      z=z_0+nt
    \end{array}
  \right.
$

\subsubsection{直线($\mathbb{R}^2$)}

$Ax+By+C=0$, 法向量$\vec{n}:\bigl[ \begin{smallmatrix} A\\B \end{smallmatrix} \bigr]$,
$s: \bigl[ \begin{smallmatrix} -B\\A \end{smallmatrix} \bigr]$

斜率为$-\frac{B}{A}$, and
$\bigl[ \begin{smallmatrix} A\\B \end{smallmatrix} \bigr]
\cdot
\bigl[ \begin{smallmatrix} x-x_0\\y-y_0 \end{smallmatrix} \bigr]
=0$

\subsection{Plane平面}
1. 过$\vec M(x_0,y_0,z_0)$, 法向量 $\vec n$(A,B,C), $\vec p$(x,y,z),平面可表示为: $t(\vec x - \vec m) $\\
$\Rightarrow t(\vec x - \vec m)\cdot \vec N = 0 \Rightarrow $
$
\bigl[ \begin{smallmatrix} x-x_0\\y-y_0\\z-z_0 \end{smallmatrix} \bigr]
\cdot
\bigl[ \begin{smallmatrix} A\\B\\C \end{smallmatrix} \bigr]
=0
\Rightarrow
A(x-x_0)+B(y-y_0)+C(z-z_0)=0
$

2. 一般式 Ax+By+Cz+D=0

3. 截距式 $\dfrac{x}{A}+\dfrac{y}{B}+\dfrac{z}{C}=0$

4. 三点式, $p_1,p_2,p_3, x_{123}$, 先求出法向量(任两个向量),用行列式, $\overrightarrow{p_1p_2} \ and \ \overrightarrow{p_2p_3}$

5. 平面束$Y$ 通过直线
$
\left \{
    \begin{array}{l}
      A_1x+B_1y+C_1z=0\\
      A_2x+B_2y+C_2z=0
    \end{array}
  \right.
$\\
$\Rightarrow
\lambda(A_1x+B_1y+C_1z) + \mu(A_2x+B_2y+C_2z)=0$, $\lambda$ and $\mu$ 不同时为0\\
$\Rightarrow A_1x+B_1y+C_1z+\lambda(A_2x+B_2y+C_2z)=0$, same as above? sure

\subsection{平面关系}
$A_1x+B_1y+C_1z+D_1=0$ and $A_2x+B_2y+C_2z+D_2=0 $

1. 平行 $n_1 \parallel n_2$: $A_1/A_2=B_1/B_2=C_1/C_2=\lambda \neq D_1/D_2 $

2. 垂直 $n_1 \perp n_2$: $ A_1A_2+B_1B_2+C_1C_2=0 $

3. $\theta$: $\vec n_1 \cdot \vec n_2 = |n_1||n_2|\cos\theta$

\subsection{直线关系(same as 平面)}
1. 平行 $\vec s_1 \parallel \vec s_2$

2. 垂直 $\vec s_1 \perp \vec s_2$

3. $\theta$:

\subsection{直线与平面}
1. 平行: $\vec s \perp \vec n \Rightarrow$
$
\bigl[ \begin{smallmatrix} A\\B\\C \end{smallmatrix} \bigr]
\cdot
\bigl[ \begin{smallmatrix} l\\m\\n \end{smallmatrix} \bigr]
=0$

2. 垂直: $\vec s \parallel \vec n \Rightarrow A/l=B/m=C/n$

3. $\theta$: $\vec l \cdot \vec s = |l||n|\cos(\pi/2 -\theta)=|l||s|\sin(\theta)$

\subsection{距离}
\textbf{Key:两条直线的夹角,可以点积也可以叉积,$\sin\theta$ and $\cos\theta$ convertable}

$\vec s \cdot \vec l = |\vec s||\vec l|\cos\theta$两边取模可求$\cos$, $\vec s \times \vec l= |\vec s||\vec l|\sin(\theta)n$两边取模可求$\sin$,可验证,取模是因为要取锐角\\
Assume: $M_0(x_0,y_0,z_0)$, $M_1(x_1,y_1,z_1)$为点或者面上的一个点

$ L:\dfrac{x-x_{1}}{l}=\dfrac{y-y_{1}}{m}=\dfrac{z-z_{1}}{m}, \Pi: Ax+By+Cz=0$\\

1. 点到点

2. 点到直线

$d=|\overrightarrow{M_0M_1}|\sin\theta=\frac{|\overrightarrow{M_0M_1}\times\vec s|}{|s|} $\\
$d=\dfrac{|Ax_0+By_0+C|}{\sqrt{A^2+B^2}} \ in \ \mathbb{R}^2,\quad L:Ax+By+C=0$

3. 点到平面

$d=|M_0M_1|\cos\langle\overrightarrow{M_0M_1},\vec n \rangle = \dfrac{|\overrightarrow{M_{0}M_{1}}\cdot \vec n|}{|\vec n|}$
$=\dfrac{|Ax_0+By_0+Cy_0+D|}{\sqrt{A^2+B^2+C^2}}$

4. 平行线、平行平面

$d=\dfrac{|D_1-D_2|}{\sqrt{A^2+B^2+C^2}}$,(while in $\mathbb{R}^{2}= \dfrac{|C_1-C_2|}{\sqrt{A^2+B^2}})$,注意NOT $||D_1|-|D_2||$

\subsection{曲线的切线、曲面的切平面}
Let 曲线 $\Gamma:
\left \{
    \begin{array}{l}
      x=\varphi(t)\\
      y=\psi(t)\\
      z=\omega(t)
    \end{array}
  \right.
$

Let 曲面 $\Omega: F(x,y,z)=0$(\textbf{Not $z=F(x,y)$, 注意和多元函数极值的区别})

曲线的切线的方向向量s(\textbf{线对线}): $(\varphi(t_0),\psi(t_0),\omega(t_0))$,
切线:$\frac{x-x_0}{\varphi'(t_0)}=\frac{y-y_0}{\psi'(t_0)}=\frac{z-z_0}{\omega'(t_0)} $\\
曲面的切平面的法向量n(\textbf{面对面}): $(F'_x(x_0,y_0,z_0),F'_y(x_0,y_0,z_0),F'_z(x_0,y_0,z_0))$\\

E.g.
$z=x^2+y^2$切平面的法向量, hint\footnote{$F(x,y,z)=x^2+y^2-z$}

\newpage
\section{不等式}
算术-几何不等式

算术平均值 $A_n=(x_1+x_2+\cdots+x_n)/n$\\
几何平均值 $G_n=\sqrt[n]{x_1x_2\cdots..x_n}$\\
$A_n \geq G_n$, $x_n \in \mathbb{R}_+$,又称均值不等式\\
证明方法:柯西逆向归纳;归纳;对数函数-琴生不等式

$1^3+2^3+3^3+\cdots+n^3 = (1+2+3+\cdots+n)^2 = [\frac{n(n+1)}{2}]^2 $\\
$1^2+2^2+3^2+\cdots+n^2 =\frac{1}{6}n(n+1)(n+2) $\\
Prove:几何证明:面积和体积; 数学归纳法证明

%%%%%%%%%%%%%%%%%%%%%%%%%%
\newpage
\subfile{calculus}

\newpage
\subfile{statistics}

\newpage
\subfile{linear}

\newpage
\subfile{high}

\newpage
\subfile{misc}


%%%%%%%%%%%%%%%%%%%%%%%%%%

\end{document}
%%% Local Variables:
%%% mode: latex
%%% TeX-PDF-mode: t
%%% TeX-engine: xetex
%%% TeX-master: t
%%% End:
