\documentclass[math.tex]{subfiles}
\begin{document}

\section{概率统计}
\subsection{tips}
Use tree diagram or venn diagram, it will be very easy.

\textbf{ Venn diagram is usually meant for sets, when used in Probability, it must be reconsidered in your mind. \\公式类的题多用venn,叙述性的题多用tree}

\textbf{But how to describe the tree branch is the most important}

% Set the overall layout of the tree
\tikzstyle{level 1}=[level distance=2.5cm, sibling distance=2.5cm]
\tikzstyle{level 2}=[level distance=2.5cm, sibling distance=2cm]

% Define styles for bags and leafs
\tikzstyle{bag} = [text width=4em, text centered]
\tikzstyle{end} = [circle, minimum width=3pt,fill, inner sep=0pt]

% The sloped option gives rotated edge labels. Personally
% I find sloped labels a bit difficult to read. Remove the sloped options
% to get horizontal labels.
\begin{tikzpicture}[grow=right, sloped]
\node[bag] {20黄 30白}
    child {
        node[bag] {white}
            child {
                node[end, label=right:
                    {white}] {}
                edge from parent
                node[above]  {$\frac{29}{49}$}
            }
            child {
                node[end, label=right:
                    {yellow}] {}
                edge from parent
                node[above]  {$\frac{30}{49}$}
            }
            edge from parent
            node[above] {$\frac{30}{50}$}
            % node[below]  {$\frac{4}{7}$}
    }
    child {
        node[bag] {yellow}
        child {
                node[end, label=right:
                    {white}] {}
                edge from parent
                node[above]  {$\frac{30}{49}$}
            }
            child {
                node[end, label=right:
                    {yellow}] {}
                edge from parent
                node[above]  {$\frac{29}{49}$}
            }
        edge from parent
            node[above]  {$\frac{20}{50}$}
    };
\end{tikzpicture}

树形图的每个分支\textbf{不必为同一属类},\textbf{垂直相加(note must include all father node)}为1
\subsection{排列、组合}
$P_n^m=n(n-1)(n-2)\cdots(n-m+1)=\frac{n!}{(n-m)!}$

$C_n^m=\frac{P_n^m}{P_m^m}=\frac{P_n^m}{m!}=\frac{n!}{(n-m)!m!}$

e.g.
0,1,2,3,4,5,6,7 排成三位数. hint\footnote{1. loci: 7x7x6; 2.$P_8^3-P_7^2$; 3. loci $P_7^1  P_7^2$}

1,1,1,2,2,3(可用来理解$C$) 排六位数. hint\footnote{ $P_6^6/3!2!$}

捆绑法:5男3女,3女必须在一起. hint\footnote{ =$P_6^6P_3^3$}

插空法:8学生4老师,老师不能相邻,且学生在中间. hint\footnote{-0-0-0-0-0-0-0-0- =$P_8^8P_7^4$}

隔板法: n元素,m堆,每堆至少一个. hint\footnote{相当于m-1个木板插入n-1个空, $C_{n-1}^{m-1}$}

\subsection{Probability}
全概率公式:$P(B)=\sum_{i=1}^nP(A_i)P(B|A_i)$, $A_1\cdots A_n$为一个完备事件组

So we have: $P(B)=P(A|B)P(B)+P(A|\bar{B})P(\bar{B})$

贝叶斯定理(Bayes' theorem): $P(A|B)=\frac{P(B|A)P(A)}{P(B)}$

So we have: $P(AB)=P(A|B)P(B)=P(B|A)P(A)$

Visualizing bayes theorem\footnote{https://oscarbonilla.com/2009/05/visualizing-bayes-theorem/} \\
$P(A|B)=\frac{|AB|}{|B|}=\frac{\frac{|AB|}{|U|}}{\frac{|B|}{|U|}}=\frac{P(AB)}{P(B)}$, that means take B as Universe(U)

$P(A-B)=P(A \bar B)=P(A)-P(AB)$\\
$P(A \cup B)=P(A)+P(B)-P(AB)$

\subsection{独立和互斥}
独立和互斥是两个概念

\textbf{A和B独立(independent) $\Leftrightarrow$ $P(AB)=P(A)P(B)$ }

SO, when independent $P(A|B)=P(A)$ and $P(B|A)=P(B)$

独立情况下venn图大多是AB交叉的\\
互斥情况下venn图是不交叉的

\subsection{概率空间、分布函数、期望、方差}
\textbf{概率空间$(\Omega, F, P)$}:是概率论的基础,概率的严格定义基于此概念

\textbf{分布函数}:$F(x)=P\{X \leq x \}=\int_{-\infty}^xf(t)dt, -\infty < x < \infty$

$F'(x)=f(x)$

\textbf{期望}(大数定律)与平均值: 期望是样本趋于无穷的极限\\
例:掷色子(可以用来理解E,离散型和连续型), 2,2,2,6,4 average=(2+2+2+6+4)/5=3\\
而E是固定的=$1\cdot \frac{1}{6} +2\cdot \frac{1}{6}+\cdots+6\cdot \frac{1}{6}=3.5$,但是3.5数值是扔不出的

\begin{align*}
E(X) &=\int_{\Omega}XdP = \left\{\begin{array}{ll}
              \sum x_ip_i & \mbox{离散函数 \textbf{Win XP}}\\
              \int_{-\infty}^{\infty}xf(x)dx & \mbox{连续函数}
            \end{array}\right.\\
E(X^2) &=\int_{\Omega}X^2dP = \left\{\begin{array}{ll}
              \sum x_i^2p_i & \mbox{离散函数}\\
              \int_{-\infty}^{\infty}x^2f(x)dx & \mbox{连续函数}
            \end{array}\right.
\end{align*}

\begin{align*}
  x=y=1
  a-b=b-c=a
\end{align*}


\textbf{方差(Variance)}: 描述它的“离散程度”,变量离“期望值”的距离\\
例:色子 2,3,3,4,4, $D=[(2-3.5)^2+(3-3.5)^2+(3-3.5)^2+(4-3.5)^2+(4-3.5)^2]/5$


\[DX=E[(X-EX)^2]=E(X^2)-(EX)^2=\left\{\begin{array}{ll}
              \sum (x_i-EX)^2p_i & \mbox{离散}\\
              \int (x-EX)^2f(x)dx & \mbox{连续}\end{array}\right.\]

$DX=E(X-\mu)^2=E(X^2-2\mu X+\mu^2)=E(x^2)-2\mu EX+(EX)^2, \mu=EX$

Why方差不用绝对值, hint\footnote{1.positive2.放大差异3.$x^2$是光滑函数,绝对值不可微4.polynomial,人们对polynomial研究的多,绝对值不是,比如Tailor}

期望值$E$是线性函数\\
$E(aX+bY+C)=aE(X)+bE(Y)+C$\\
$E(XY)=EX \cdot EY$ (XY的协方差为0(此时称不相关,两个随机变量独立是一种情况))\\
$E(E(X))=E(X)$\\
$E(X-E(X))=0$\\
$E(g(X))\neq g(E(x))$一般情况下

$D(C)=0$\\
$D(aX)=a^2DX$\\
$D(aX+C)=a^2DX$\\
$D(X \pm Y)=DX+DY \pm 2E\{(X-EX)(Y-EY)\}$

e.g.

Find $E(X^2)$, $x \in \{1,2,3,4,5,6\}$, hint1\footnote{$x \in \{1,2,3,4,5,6\}$,x$^2 \in \{1,4,9.16.35,36\}$,$E(X^2)=\sum X_i^2P_i\{X=x_i\}=1/6+4/6+9/6+16/6+36/6=15.167$}, hint2\footnote{let $f(x)=x^2$, then $E(x^2)=\sum (x_i)^2P(X=x_i)$}

$X\sim B(100,0.2)$,求$D(X^2)$, Note: not the $n\times(0,1)$分布,概率空间变了,即$E(g(X))\neq g(E(x))$一般情况下


\subsubsection{变异系数(离散系数) coefficient of variation}
$c_v=\frac{\sigma}{\mu}$,无量纲量,概率分布离散程度的归一化量度,变异系数也称为标准差离率或单位风险,只在平均值不为0时有意义,一般适用于大于零的情况。

变异系数只对由比率标量计算出来的数值有意义。举例来说,对于一个气温的分布,使用开尔文或摄氏度来计算的话并不会改变标准差的值,但是温度的平均值会改变,因此使用不同的温标的话得出的变异系数是不同的。也就是说,使用区间标量得到的变异系数是没有意义的

\subsection{常见分布}
\subsubsection{0-1、Bernulli Trial}
同条件下重复、相互独立的试验,只有两种结果,发生或者不发生,如掷硬币\\
$EX=\mu=\sum X_iP(X=x_i)=1\cdot p +0\cdot (1-p)=p$\\
$DX=\sigma^2=(1-p)^2p+(0-p)^2(1-p)=p(1-p)$

\subsubsection{二项分布}
N次独立的Bernulli Trial, 记为 $X \sim B(n,p)$\\
N次实验中正好得到k次成功的概率为$C_n^kp^k(1-p)^{n-k}$

$P\{X=k\}=C_n^kp^k(1-p)^{n-k}$\\
$E=\sum_{i=1}^n\mu=np$\\
$D=\sum_{i=1}^n\sigma^2=np(1-p)$

e.g.\\
$B(2,p)$, if $p\{X\geq1\}=5/9$, find the p and DX, hint\footnote{1.$P\{X\geq 1\}=P\{X=1\}+P\{X=2\}$
2. $np(1-p)$
}

\subsubsection{泊松分布}
$P_k=\frac{\lambda^k}{k!}e^{-\lambda}$\\
$E=D=\lambda$, hint\footnote{
$
E=\sum_{k=0}^{\infty}kP_k=e^{-\lambda}\sum_{k=0}^{\infty}k\frac{\lambda^k}{k!}=e^{-\lambda}\sum_{k=1}^{\infty}\frac{\lambda^k}{(k-1)!}=e^{-\lambda}\lambda \sum_{k=0}^{\infty}\frac{\lambda^k}{k!}=e^{-\lambda}\lambda e^{\lambda}=\lambda
$}

\subsubsection{指数分布}
$P(x)=\lambda e^{-\lambda x}, x\geq 0$\\
$E=1/\lambda$\\
$D=1/\lambda^2$

\subsubsection{均匀分布}
$f(x)=\frac{1}{b-a}, F(x)=\frac{x}{b-a}$

x落在任一子区间的概率只与长度有关,而与位置无关。具有下属意义的等可能性??

$EX=\int_a^bxf(x)dx=\frac{a+b}{2}$\\
$DX=E(X^2)-(EX)^2=\frac{(a-b)^2}{12}$, if you use $D=\int_a^b(x-EX)^2f(x)dx$很繁琐

\subsubsection{正态分布、高斯分布}
众数=中位数=E=$\mu$

标准正态分布 $\mu=0$ $\sigma^2=1$

\subsection{Examples}

A和B独立,$P(A \cup B)=0.8, P(A)=0.4, P(\bar B|A)=?, hint$\footnote{$P(A\cup B)=P(A)+P(B)-P(A)P(B)$ answer is $1/3$ }

共50球,20黄球,30白球,两人依次取出不放回,求第二人取黄球的P,求 如第二人黄球第一人黄球的P, 思考第一人黄球第二人也是黄球的P, hint\footnote{Tree diagram, final answer is 2/5, 19/49, 19/49}

参加考试考生中,本专业学生占6成,本专业通过率是85\%,非本专业通过率是50\%。某位考生通过了考试,他是本专业的P是多少, hint\footnote{重点是怎么画tree, tree的各节点是同性质的吗??, 考生\{考生-本专业\{考生-通过、考生-未通过\},考生-非本专业\{考生-通过、考生-未通过\}\}, answer is 51/71}

10个产品,4次品,任取两个(第一次第二次),至少一个正品的P是多少,hint\footnote{1. 1-$\frac{C_4^2}{C_{10}^2}$ 2. Tree diagram}

快递员,A到B地送货,开汽车或骑电动车,分别记录了deliver cargo 100times 的时间,开汽车:平均24分钟,方差为36;电动车:平均34分钟,方差为4\\
quest1: 建议用哪种方式,and why\\
question2: 如果开车和电动车的送货时间都服从正态分布,如果某次送货有38分钟可用,应该选哪种方式,如果有34分钟可用,选哪种。\\
hint\footnote{key1:变异系数?key2正态分布的转换到标准正态分布$N(0,1)$\\
  1.$c_1=6/24, c_2=2/34$, 表示离散程度,变异系数越小,分布越集中,所以选电动车。\\
  2.$X~N(24,36)$, $Y~N(34,4)$,标准化$P(X\leq38)=P(\frac{X-24}{6}\leq \frac{38-24}{6})=\phi(7/3)$, 另一个是$\phi(2)$, 故选汽车,选概率大的??
}

甲乙两种饮料,颜色气味很相似,饮料放在外观相同的6个杯子中,每个品牌3个杯子作为实验样品。从6个杯子中随机选3杯作为一次实验,若所选饮料全部为甲,则视为成功。独立进行5次实验,求3次成功的P,
hint\footnote{一次实验成功的P为$\frac{C_3^3}{C_6^3}=1/20$, 5次独立实验,$X\sim B(5,1/20)$,$P\{X=3\}=C_5^3(1/20)^3(19/20)^2$}


\end{document}

%%% Local Variables:
%%% mode: latex
%%% TeX-master: "math"
%%% TeX-engine: xetex
%%% End:
