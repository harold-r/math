\documentclass[math.tex]{subfiles}
\begin{document}

\section{线性代数 Linear algebra}

\subsection{Vector}
\subsubsection{Basic}

The fundamental for linear algebra is the vector

Two fundamental vector operations: additon and scalar multiplication

basic vectors: $\hat{\imath}$ and $\hat{\jmath}$

"Span" is the set of all linear combinations: $a\vec{v}+b\vec{w}$

linearly independent:

向量正统的表示为 $\bigl[ \begin{smallmatrix}  x\\y \end{smallmatrix} \bigr]$,有时候(x,y), $(x,y)^{T}$, $[x,y]^T$

vector in $\mathbb{R}^2$ coordinate is $\bigl[ \begin{smallmatrix} x\\y \end{smallmatrix} \bigr]$

vector in $\mathbb{R}^3$ coordinate is $ \bigl[ \begin{smallmatrix} x\\y\\z \end{smallmatrix} \bigr]$ and so on and on

When we say $\vec a$ 可以任意位置,座标系内也可以任意位置,但如果用tuple表示(tuple就是坐标),那么必须是从原点开始的。

||x|| 范数:Norm, scalar\\
|x| 绝对值,是实数集上的一个范数,一维向量中 ||x|| = |x| \\
$ x = (x_1,x_2,x_3...x_n)^T \Rightarrow ||x||= \sqrt{x_1^2+x_2^2+x_3^2+...+x_n^2} $\\
欧几里得空间里,内积等价于点积,故:||x||=$\sqrt{x \cdot x}$

\subsubsection{Matrix as linear transformation}

Matrix is the transformations of space

linear tranformations are  a way to move around space such that the grid lines remain parallel and evenly spaced and such that the origin remains fixed

"linear" has two properties: 1.line remain lines, 2. origin remains fixed

e.g. $90^{\circ}$ rotation counterwise:

$\hat{\imath} \quad \bigl[ \begin{smallmatrix} 1\\0 \end{smallmatrix} \bigr] \rightarrow \bigl[ \begin{smallmatrix} 0\\1 \end{smallmatrix} \bigr]$ where $\hat{\imath}$ land and
$\hat{\jmath} \quad \bigl[ \begin{smallmatrix} 0\\1 \end{smallmatrix} \bigr] \rightarrow \bigl[ \begin{smallmatrix} -1\\0 \end{smallmatrix} \bigr]$ where $\hat{\jmath}$ land, so the transormation is \\

$\begin{bmatrix}
  0 & -1 \\
  1 & 0
\end{bmatrix}
$

\subsubsection{Vector 加减}
坐标对应的加减
\subsubsection{Dot Product(点积)}
$\vec{a} \cdot \vec{b}$ = |a||b|$\cos\theta$, |a|表示模 \\
代数定义: $\vec{a} \cdot \vec{b} = \sum_{i=1}^na_ib_i = a_1b_1+a_2b_2+...+a_nb_n$\\
几何定义:投影 and 余弦定理(why $\cos$) $(\vec{a} - \vec{b})^{\,2}= \vec{a}^{\,2}+\vec{b}^{\,2}-2|a||b|\cos\theta $
\subsubsection{Cross Product(叉积)}
$\vec{a} \times \vec{b} = \|a\|\|b\| \sin(\theta)n$; n is normal \underline{\textbf{unit}} vector\\
几何意义:平行四边形的面积\\
右手定则 $\vec{a} \times \vec{b} = -(\vec{a} \times \vec{b})$\\
If $ U = (u_1,u_2,u_3), V = (v_1,v_2,v_3)$, (i,j,k)为基向量\\
$
U \times V =
\begin{vmatrix}
  i & j & k\\
  u_1 & u_2 & u_3\\
  v_1 & v_2 & v_3
\end{vmatrix}
=
\begin{vmatrix}
  i & u_1 & v_1\\
  j & u_2 & v_2\\
  k & u_3 & v_3
\end{vmatrix}
=
i\bigl[\begin{smallmatrix} u_2&u_3\\v_2&v_3 \end{smallmatrix} \bigr]-
j\bigl[\begin{smallmatrix} u_1&u_3\\v_1&v_3 \end{smallmatrix} \bigr]+
k\bigl[\begin{smallmatrix} u_1&u_2\\v_1&v_2 \end{smallmatrix} \bigr]
$

\subsubsection{Vector Relations(向量关系)}
向量平行$ \vec{a}=\lambda\vec{b} \Leftrightarrow \frac{x_1}{x_2}=\frac{y_1}{y_{2}}=\frac{z_1}{z_2} = \lambda $\\
向量垂直$ \vec{a}\cdot\vec{b}=0 \Leftrightarrow x_1x_2+y_1y_2+z_1z_2=0  $

\subsubsection{Linear Independence}
$c_1v_1+c_2v_2+\cdots + c_nv_n=0$, 有非全零的$c_1, c_2, \cdots c_n$则线性相关 dependent,否则independent\\
即:有一个向量可以用其他向量组合来表示是线性相关

Basis of a subspace. Basis is “minimum” set of vectors that spans the subspace\\
$V=span(\vec v_1, \vec v_2 \cdots \vec v_n)$. if V is a basics, ${\vec v_1 \cdots }$ are linear indenpendent.\\
$S=\{[2,3]^T,[7,0]^T\}$, $span(S)=\mathbb{R}^2$

e.g.\\
$S= (1,-1,2), (1,1,3), (-1,0,2)$, $span(S)=\mathbb{R}^3?$ and linear independent?

Prove: $\mathbb{R}^4$内,$[1,4,2,-3]^T, [7,10,-4,-1]^T,[-2,1,5,-4]^T$是线性相关的。hint\footnote{Using defination: let $k_1\alpha_1+k_2\alpha_2+k_3\alpha_3=0 \Leftrightarrow [\alpha_1,\alpha_2,\alpha_3]\bigl[ \begin{smallmatrix} k_1\\k_2\\k_3 \end{smallmatrix} \bigr]=0$}

\subsubsection{MISC}
Cauchy–Schwarz inequality:
$\vec a, \vec b \in \mathbb{R}^n$, $|\vec a \cdot \vec b| \leq \|\vec a\| \|\vec b\|$, proof

Triangle inequality: $\|\vec a+\vec b \| \leq \|\vec a\| + \| \vec b\|$, proof



\subsection{方程的解}
\subsubsection{$Ax=b$}
有\quad 解的充要条件: $r(A)=r(A|b)$\\
唯一解的充要条件:$r(A)=r(A|b)=n$\\
无穷解的充要条件:$r(A)<r(A|b)$

克莱姆法则(Cramer's Rule): $|A|\neq 0$则方程有解且唯一

基础解系(化简为行最简,k不同时为0):
\[
  \begin{bmatrix}
    1&0&-4&-5 \\
    0&1&3&4\\
    0&0&0&0
  \end{bmatrix}
  x=0 \Rightarrow
  \left \{
    \begin{array}{l}
      x_1=4x_3+5x_4\\
      x_2=-3x_3-4x_4\\
      x_3=1x_3+0x_4\\
      x_4=0x_3+1x_4
    \end{array}
  \right.
  Let \quad \alpha_1=
  \begin{bmatrix}
    4\\
   -3\\
    1\\
    0
  \end{bmatrix}
  \quad
  \alpha_2=
  \begin{bmatrix}
    5\\
    -4\\
    0\\
    1
  \end{bmatrix}
  \Rightarrow x=k_1\alpha_1+k_2\alpha_2
\]

\subsubsection{$Ax=0$}
Cramer's Rule:\\
只有零解: $|A|\neq 0$\\
有非零解: $|A|=0$

\subsection{伴随矩阵}
$adj(A)=C^T$ and $C_{ij}=(-1)^{i+j}M_{ij}$, and $M_{ij}$是余子式, C是代数余子式\\
即:先代数余子式,再转置

if $A=\bigl[ \begin{smallmatrix} a&b \\ c&d  \end{smallmatrix} \bigr]$ then $adj(A)=\bigl[ \begin{smallmatrix} d&-b \\-c&a \end{smallmatrix} \bigr]$

\subsection{逆矩阵$A^{-1}$}
可逆的充要条件: $|A|\neq 0$

\textbf{求法:}\\
1. $A^{-1}=\frac{A^{\ast}}{|A|}$, $A^{\ast}$为伴随矩阵\\
2. $(A|E) \to (E|A^{-1})$

\textbf{性质:}\\
$(\lambda A)^{-1}=\frac{1}{\lambda}A^{-1}$\\
$(AB)^{-1}=B^{-1}A^{-1}$, how to prove, hint\footnote{$ABB^{-1}A^{-1}=E$, matrix的结合律,(AB)(B$^{-1}A^{-1)}=E$, also $(AB)(AB)^{-1}=E$, so ...)}\\
$(A^T)^{-1}=(A^{-1})^T$\\
$det(A^{-1})=\frac{1}{det(A)}$

对于转置矩阵\\
$(\lambda A)^{-T}=\lambda A^{T}$
$(AB)^T=B^TA^T$, 不同于上,要用矩阵结构去证明\\

\subsection{正交、合同、正定、相似}

\textbf{正交:}$AA^T=E$ or $A^TA=E$。xx交得带T

\textbf{合同:}$\exists$可逆的$C$, $B=C^TAC$,则称A、B合同。合在一起,T

\textbf{相似:} $P^{-1}AP=B$,称为A和B相似\\
必要条件:1. 同样特征值(也意味着$|A|=|B|$). 2. rank相同 3. 迹相同(主对角线的和)

% e.g.\\
% A$\bigl[ \begin{smallmatrix} 1&4 \\ 2&3 \end{smallmatrix} \bigr],$
% B$\bigl[ \begin{smallmatrix} 6&a \\ -1&b \end{smallmatrix} \bigr]$相似,Find $a,b$, hint\footnote{
% $\left \{
%   \begin{array}{left}
%     1+3=6+b\\
%     2+4=-1+a
%   \end{array}
% \right.
% $
% }

\textbf{二次型的正定:}\\
二次型:$f(x_1,x_2\cdots x_n)=a_{11}x_1^2+2a_{12}x_1x_2+\cdots =\sum_{i=1}^na_{ii}x_i^2+2\sum_{i\neq j}a_{ij}x_ix_j$\\
$A=
\begin{bmatrix}
  a_{11}&a_{12}&\cdots&a_{1n}\\
  a_{12}&a_{22}&\cdots&a_{2n}\\
  \vdots&\vdots&\vdots&\vdots\\
  a_{1n}&a_{2n}&\cdots&a_{nn}
\end{bmatrix}
$

二次型的正定定义:$x_{1,2,\cdots ,n}$不全为0时,$f(x_1,x_2\cdots x_n)>0$\\
充要条件是:A的特征值都是正数 或者 顺序主子式$>0$ 或者合同于单位矩阵

所有特征值都是正数的矩阵被称为正定\\
所有特征值都是非负数的矩阵被称为半正定\\
所有特征值都是负数的矩阵被称为负定\\
所有特征值都是非正数的矩阵被称为半负定

例:$x^2-xy+y^2$ $\to$ $A=\bigl[ \begin{smallmatrix} 1&-\frac{1}{2}\\ -\frac{1}{2}&1 \end{smallmatrix} \bigr] $,顺序主子式大于0

\textbf{顺序主子式}
$
\begin{bmatrix}
  1&2&3\\
  4&5&6\\
  7&8&9
\end{bmatrix}
\Rightarrow
\left \{
  \begin{array}{l}
    D_1=|1|\\
    D_2=\mid \begin{smallmatrix} 1&2\\4&5 \end{smallmatrix} \mid\\
   D_3=\mid \begin{smallmatrix} 1&2&3\\4&5&6\\6&7&8 \end{smallmatrix} \mid                      \end{array}
\right.
$

\subsection{特征值、特征向量}
定义:$\exists \lambda$ and $\alpha$, 使$A\alpha=\lambda \alpha$, 则$\lambda$为特征值,而$\alpha$为特征值$\lambda$的特征向量

\textbf{特征值求法}
$A\alpha=\lambda \alpha \Rightarrow (\lambda E-A)\lambda=0$,齐次线性方程有非零解$|\lambda E-A|=0$

\textbf{特征向量求法}:$\lambda_i$为一特征值,$(\lambda_iE-A)x=0$的解即是,如果基础解系$\alpha_1,\alpha_2\cdots\alpha_s$,特征向量$=k_1\alpha_1+k_2\alpha \cdots +k_s\alpha$, $k$不同时为0

特点:\\
1. $|A|=\displaystyle\prod_{i=1}^n\lambda_i$,即所有特征值的乘积,$|A|\neq 0 \Rightarrow A$无零特征值

2. 不同的特征值对应的$\alpha$线性无关

\subsection{欧式空间、标准正交基}
定义了内积的线性空间称为欧式空间, $(\alpha, \beta)$为$\alpha$和$\beta$的内积

标准正交基: 施密特正交化(所有$\beta$化,替一个)
\[
\left \{
  \begin{array}{l}
    \beta_1=\alpha_1\\
    \beta_2=\alpha_2-\frac{(\alpha_2,\beta_1)}{(\beta_1,\beta_1)}\cdot \beta_1\\
    \beta_3=\alpha_3-\frac{(\alpha_3,\beta_1)}{(\beta_1,\beta_1)}\cdot \beta_1-
            \frac{(\alpha_3,\beta_2)}{(\beta_2,\beta_2)}\cdot \beta_2\\
    \vdots \\
    \beta_n=\alpha_n-\frac{(\alpha_n,\beta_1)}{(\beta_1,\beta_1)}\cdot \beta_1-\cdots
            \frac{(\alpha_n,\beta_{n-1})}{(\beta_{n-1},\beta_{n-1})}\cdot \beta_{n-1}
  \end{array}
\right.
\]

\subsection{MISC}

$\mid \begin{smallmatrix} A&B\\0&D \end{smallmatrix} \mid = det(A)det(D)$

$|A|=\displaystyle\prod_{i=1}^n\lambda_i$

反对称矩阵 $A^T=-A$,对角线为零,其他反

单位正交矩阵, $A$各行列为单位向量,且两两正交

像即线性映射、线性变换\\
求直线$y=3x$在矩阵$\bigl[ \begin{smallmatrix} 1&-1\\-1&1 \end{smallmatrix} \bigr]$所对应的线性变换下的像的方程。hint\footnote{随意取直线两点,比如$(0,0),(1,3)$, $\bigl[ \begin{smallmatrix} 1&-1\\-1&1 \end{smallmatrix} \bigr] \bigl[ \begin{smallmatrix} 0\\0 \end{smallmatrix} \bigr]=\bigl[ \begin{smallmatrix} 0\\0 \end{smallmatrix} \bigr]$, the other is $\bigl[ \begin{smallmatrix} -2\\2 \end{smallmatrix} \bigr]$, so 新方程经过$(0,0),(-2,2)$}


\end{document}

%%% Local Variables:
%%% mode: latex
%%% TeX-master: "math"
%%% TeX-engine: xetex
%%% End:
