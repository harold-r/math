\documentclass[../main.tex]{subfiles}
\graphicspath{{\subfix{../images/}}}

\begin{document}

\section{MISC}

$0!$规定为$1$

数域(域:field):指一个数集,它对加法、减法、乘法和除法(除数不为零)运算是封闭的,并且包含$0$和$1$\\
换句话说,对数域中任意两个数进行这四种基本运算,其结果仍然属于这个数域。\\
常见的数域包括有理数域(Q)、实数域(R) 和复数域(C)\\
一个集合成为数域,需满足以下条件:\\
1. 包含0和1:数域中必须有加法单位元0和乘法单位元1\\
2. 封闭性:加法和减法封闭;乘法封闭;除法封闭\\
非数域例子:自然数集和整数集,不构成数域,因为除法运算不封闭,例如$2/3$不属于自然数或者整数


\section{代数基本定理(Fundamental theorem of algebra)}
Also called "d'Alembert–Gauss theorem"\\
描述为:任何一个复系数的一元$n$ 次多项式方程($n\geq 1$),至少有一个复数根。

有时候这个定理描述为:任何一个非零的一元n次复系数多项式,都正好有$n$个复数根(重根视为多个根)。但实际上,是“至少有一个根的”直接结果,因为把多项式除以它的线性因子可以推出。也就是说,任何一个$n$次多项式,都可以因式分解为$n$个复系数一次多项式的乘积(根据多项式除法\todo{Proof?})。

推论:任何一个非零的一元n次复系数多项式,都正好有n个复数根(重根视为多个根)。

意义:复数域是代数封闭的;该定理是代数学和近世代数中的一个基础性结论

尽管这个定理被命名为“代数基本定理”,但它还没有纯粹的代数证明,许多数学家都相信这种证明不存在。另外,它也不是最基本的代数定理;因为在那个时候,代数基本上就是关于解实系数或复系数多项式方程,所以才被命名为代数基本定理。

所有的证明都包含了一些数学分析,至少是实数或复数函数的连续性概念。有些证明也用到了可微函数,甚至是解析函数。



\section{多项式除法、多项式余式定理(Polynomial division, Polynomial remainder theorem)}

\begin{flalign*}
  \frac{P(x)}{D(x)} = Q(x)+\frac{R(x)}{D(x)} \Rightarrow P(x)=D(x)Q(x)+R(x)&&
\end{flalign*}

If $D(x)=x-a$, then $P(x)=(x-a)Q(x)+R(x) = (x-a)Q(x)+r$\\
根据定义,$R(x)$的次数小于$1$, so $R(x)$只能为常数\\
$\Rightarrow P(a)=(a-a)Q(x)+r=r$\\
得到\textbf{多项式余式定理}:多项式$P(x)$除以$x-a$所得的余式$=P(a)$

dividend = divisor x quotient + reminder

\subsection{Examples}
Let $f(x)=x^{3}-12x^{2}-42$, divided by $x-3$, gives the quotient $x^{2}-9x-27$, and the remainder $-123$.\\
By the polynomial remainder theorem, $f(3)=-123$

寻找多项式的切线? \todo{?直觉要用微积分,但是这个是啥情况?} \url{https://zh.wikipedia.org/wiki/%E5%A4%9A%E9%A1%B9%E5%BC%8F%E9%99%A4%E6%B3%95}


\section{因式定理(Factor theorem)}
The Factor theorem connects polynomial factors with polynomial roots.(关于多项式的因式和零点的定理)

一个多项式$f(x)$有一个因式$ax-b$当且仅当 $f(\frac{b}{a})=0$

普遍应用于因式分解,利用长除法,除以零点$(x-a)$

Example:

分解因式:$(x-y)^{3}+(y-z)^{3}+(z-x)^{3}$\\
$x=y, y=x, x=z$是$0$点,so $k(x-y)(y-z)(x-z)$, let $x=0,y=1,z=2 \Rightarrow k=3$

\section{因式分解}
\subsection{因式分解定理}
数域$F$上的每个次数$\ge 1$的多项式$f(x)$都可以分解为数域$F$上一些不可约多项式的乘积,并且是唯一的,即:\\
$f(x)=p_{x}(x)p_{2}(2)p_{3}(x)\cdots p_{s}(x) = q_{1}(x)q_{2}(x)q_{3}(x)\cdots q_{t}(x)$,其中$p_{i}(x)$和$q_{j}(x)$都是数域$F$上的不可约多项式,那么必有$s=t$,而且可以适当排列因式的次序,使得\\
$p_{i}(x)=c_{i}q_{i}(x)$

\subsection{分解方法}
公因式

公式法

分组分解

拆添项

十字交叉

一次因式检验法(有理根定理)


\section{算术基本定理}


\section{有理根定理(Rational root theorem)}
Also called rational root test, rational zero theorem, rational zero test or $p/q$ theorem

描述:对于$a_{n}x^{n} + a_{n-1}x^{n-1}+\cdots + a_{1}x+a_{0}=0$, \colorbox{BurntOrange}{系数$a_{i}\in \mathbb{Z}$}, and $a_{0}, a_{n} \neq 0$.\\
该定理指出,如果存在有理根$x=\frac{p}{q}$, written in lowest term(that is $p$ and $q$ are relatively prime, 互质),满足:\\
$p$是$a_{0}$的整数因子,i.e. $p|a_{0}$.整除符号,Tips\footnote{$a|b$: $a$整除$b$,$b$能被$a$整除,也就是$b$除以非零$a$,商是一个整数. i.e. $2|6$}\\
$q$是$a_{n}$的整数因子,i.e. $q|a_{n}$.

该定理是高斯定理关于多项式分解的一个特例

\subsection{Proof}
Let $P(x) = a_{n}x^{n}+a_{n-1}x^{n-1}+\cdots +a_{1}x+a_{0}$ with $a_{i} \in \mathbb{Z}$, $a_{0},a_{n} \neq 0$

Suppose $P(p/q) = 0$ for some coprime $p,q \in \mathbb{Z}$:\\
$P(\frac{p}{q}) = a_{n}(\frac{p}{q})^{n} + a_{n-1}(\frac{p}{q})^{n-1}+\cdots + a_{1}(\frac{p}{q}) + a_{0} = 0$\\
$\Rightarrow a_{n}p^{n}+ a_{n-1}p^{n-1}q+\cdots +a_{1}pq^{n-1}+ a_{0}q^{n} = 0$\\
$\Rightarrow p(a_{n}p^{n-1}+a_{n-1}qp^{n-2}+\cdots +a_{1}q^{n-1}) = -a_{0}q^{n} \Rightarrow ()=-a_{0}\frac{q^{n}}{p}$ \\
$\Rightarrow q(a_{n-1}p^{n-1}+a_{n-2}qp^{n-2}+\cdots +a_{0}q^{n-1}) = -a_{n}p^{n} \Rightarrow ()=-a_{n}\frac{p^{n}}{q}$\\
我们注意到,括弧内是整数,因为$a_{i}$是整数,所以这是关键

$p,q$互质,$\frac{p}{q} = \pm \frac{a_{0}\text{的因子}}{a_{n}\text{的因子}}$

注意$p, q$为$1$的特殊情况,显而易见$1$永远是第一个选择

关键点:\\
1. 系数是整数\\
2. 如果存在有理根,则必符合此定理,否则存在无理根(如$\sqrt{89}$)亦或者复数根

\subsection{Example}
$x^{3}-7x+6=0$: \\
有理根有可能是:$\pm \frac{\{1,2,3,6\}}{1}=\pm {1, 2, 3, 6}$,恰好$1, 2, -3$,所以也恰好可以写为:$(x-1)(x-2)(x+3)=0$

$3x^{3}-5x^{2}+5x-2=0$, 如果有有理根,则必在$\pm \frac{1,2}{1,3}=\pm{1,2, \frac{1}{3}, \frac{2}{3}}$中。8个候选根,需要测试8次,最后才知道$x=2/3$是唯一有理根.\\
很是繁琐不是?所以可以通过评估$P(r)$来测试缩小范围(比如使用秦九韶算法?)。\\
Firstly, if $x<0$, the $P$ will be negative, so every root is positive\\
$P(1)=1$, so 1 is not the root. Moreover, if one sets $x=1+t$, so $Q(t)=P(1+t)$, 展开后,三次项是3,一次项是1,implies $Q$ must belongs to ${\pm 1, \pm \frac{1}{3}}$, and $P$ satisfy $x=1+t \in {2,0,4/3,2/3}$. 再次显示必须为正,两个候选项是$2, 2/3$, 将$2$带入,显然不是,最后测试$2/3$\

If $a, b$ and $\frac{a^{2}}{b}+\frac{b^{2}}{a}$ are integers, then both $\frac{a^{2}}{b}$ and $\frac{b^{2}}{a}$ must be integers.\\

% see https://en.wikipedia.org/wiki/Rational_root_theorem

\section{韦达定理(Vieta's formulas)}
Any general polynomial of degree $n, P(x)=a_{n}x^{n}+ a_{n-1}x^{n-1}+\cdots +a_{1}x+a_{0}$, by the "fundamental theorem of algebra", roots are $x_{1}, x_{2}, x_{3}\cdots$
\[
  \left\{
    \begin{array}{l}
      x_{1}+x_{2}+x_{3}+\cdots + x_{n-1}+ x_{n} = -\frac{a_{n-1}}{a_{n}}\\
      (x_{1}x_{2}+x_{1}x_{3}+x_{1}x_{4}+\cdots +x_{1}x_{n})+(x_{2}x_{3}+x_{2}x_{4}+\cdots +x_{2}x_{n})+\cdots + x_{n-1}x_{n} = \frac{a_{n-2}}{a_{n}}\\
      \vdots\\
      x_{1}x_{2}x_{3}\cdots x_{n} = (-1)^{n}\frac{a_{0}}{a_{n}}
    \end{array}
  \right.
\]

\subsection{Proof}
$a_{n}x^{n}+a_{n-1}x^{n-1}+\cdots + a_{1}x + a_{0} = a_{n}(x-x_{1})(x-x_{2})\cdots (x-x_{n}) $\\
展开后比较系数\\
\[
  \left\{
    \begin{aligned}
      a_{n-1} &= -a_{n}(x_{1}+x_{2}+\cdots + x_{n-1}+ x_{n})\\
      a_{n-2} &= a_{n}[(x_{1}x_{2}+x_{1}x_{3}+\cdots +x_{1}x_{n})+ (x_{2}x_{3}+ x_{2}x_{4}+\cdots + x_{2}x_{n})+\cdots + x_{n-1}x_{n}]\\
      \vdots \\
      a_{0} &= (-1)^{n}a_{n}x_{1}x_{2}\cdots x_{n}
    \end{aligned}
  \right.
\]

\subsection{韦达定理的逆定理}
对于一元二次方程

利用圆来研究一元二次方程?  http://202.175.82.54/tplan/2006/intro/R027.pdf


\subsection{Examples}
If $n=2$(quadratic), $ax^{2}+bx+c=0 = a(x-x_{1})(x-x_{2})$展开比较即有,也可以用求根公式

if $n=3$, $x_{1}, x_{2}, x_{3}$是$ax^{3}+bx^{2}+cx+d=0$的三个根, then:\\
$ax^{3}+bx^{2}+cx+d = a(x-x_{1})(x-x_{2})(x-x_{3})=a(x^{3}-(x_{1}+x_{2}+x_{3})x^{2}+(x_{1}x_{2}+x_{2}x_{3}+x_{1}x_{3})x- x_{1}x_{2}x_{3}=0)$, That is\\
$x_{1}+x_{2}+x_{3}= -\frac{b}{a}, x_{1}x_{2}+x_{1}x_{3}+x_{2}x_{3}=\frac{c}{a}, x_{1}x_{2}x_{3}=-\frac{d}{a}$


\section{排列组合(permutation and combination)}

\subsection{排列(permutation)}
Permutaion(排列、变换、置换,比如古典密码里的置换) or Arrangement,所以数学符合$P$和$A$都可以。

利用乘法原理:$A_{n}^{k}=\overbrace{n(n-1)(n-2)...(n-k+1)}^{\text{k factors}}=\frac{n!}{(n-k)!}$\\
Also use:  $P^{n}_{k}$, $P(n,k)$, $_{n}P_{k}$, $^{n}P_{k}$, $P_{n,k}$. Note the slight difference: $P^{n}_{k}$ and $A_{n}^{k}$

重复排列:从$n$个元素中取出$k$个元素,$k$个元素可以重复:$U^{n}_{k}=n^{k}$

\subsection{组合(combination)}
Combination just likes permutation, but the order doesn't matter

This formula can be derived from the fact that each k-combination of a set $S$ of $n$ members has permutations so\\
$P_{n}^{k}=C_{n}^{k}\times k!$ or $C_{n}^{k}=P_{n}^{k}/k!$. The $C_{n}^{k}$ often denoted by $\binom{n}{k}$

\section{二项式定理(Binomial theorem)}
\[
  (x+y)^{n}=C_{n}^{0}x^{n}y^{0}+C_{n}^{1}x^{n-1}y^{1}+C_{n}^{2}x^{n-2}y2+\cdots +C_{n}^{n}x^{0}y^{n}
\]

Examples:

$(x+y)^{0}=1$\\
$(x+y)^{1}=x+y$\\
$(x+y)^{2}=x^{2}+2xy+y^{2}$\\
$(x+y)^{3}=x^{3}+3x^{2}y+3xy^{2}+y^{3}$\\
$(x+y)^{4}=x^{4}+4x^{3}y+6x^{2}y^{2}+4xy^{3}+y^{4}$

$(x+y)^{3}=xxx+xxy+xyx+xyy+yxx+yxy+yyx+yyy$, total $2^{3}$ terms

Let $x=y=1$, we have $2^{n}=C_{n}^{0}+C_{n}^{1}+\cdots +C_{n}^{n}$

\subsection{Proof}

Method 1:数学归纳法(inductive proof)

Method 2: 组合方法

$(a+b)^{n}=\overbrace{(a+b)(a+b)\cdots (a+b)}^{\text{n terms}}$, $n$个括号相乘,从$n$个选出$k$个括号中的$a$,再从剩余的$n-k$个括号中选出$(n-k)$个$b$,得到一组$a^{k}b^{n-k}$,而这种选法共有$C_{n}^{k}$种,故总共有$C_{n}^{k}$个$a^{k}b^{n-k}$;其他同理

\subsection{More}
$(1+x)^{-1}$

$(1+x)^{\frac{1}{2}}$

$(1+\frac{1}{n})^{n}$

Multinomial theorem:\\
$(x_{1}+x_{2}\cdots +x_{m})^{n}$


\section{绝对值}
\subsection{绝对值的意义}
本质是表示距离,比如数轴上的线段距离,差值的绝对值

第一要务一般是如果去绝对值,从代数上看就是要去讨论

不但要会去绝对值,还要会用绝对值列出题目相应的等式或者不等式,然后去解

又比如$|3-2x| + |x-3|$的最小值,要善于变换,以方便几何上的直观

\subsection{最值}
$f(x)= |x+1|+ |2-x|$的最值\\
$f(x) = |x+1| + |2x-1|$的最值,Tips\footnote{$|2x-1| = 2|x-\frac{1}{2}|$}\\
$f(x)= |x+1| + |x| + |x-2|$的最值\\
$f(x)=|2x-1|+|4x-3|$\\
$f(x)=||x-1|-3|$


\subsection{推广}
$f(x) = |x-a_{1}| + |x-a_{2}|+ |x-a_{3}| + \cdots + |x-a_{n}| $
$f(x)=|x+1|+|2x-1|$可以化简为:$f(x)=|x+1|+2|x-\frac{1}{2}|=|x+1|+|x-\frac{1}{2}|+|x-\frac{1}{2}|$

奇点偶段,证明方法:从1到3到5,从2到4到6,以至无穷


\section{整除规则(divisibility rule)}
Let $A$ = $\overline{a_{n}a_{n-1}...a_{2}a_{1}} = a_{n}\times 10^{n-1}+a_{n-1}\times 10^{n-2}...+a_{2}\times 10 + a_{1}$
\subsection{基本判别(rules)}

被2整除:\\
The last digit is even

被3整除和被9整除:\\
The sum of digits must be divisible by $3$ or $9$.
Tips\footnote{$A = a_{n}\times (9+1)^{n-1}+a_{n-1}\times (9+1)^{n-2}...+a_{2}\times (9+1)+a_{1} \\
~~~   = (a_{n}\times 9^{n-1} + a_{n-1}\times 9^{n-2}+...+a_{2}\times 9)+(a_{n}+a_{n-1}+...+a_{2}+a_{1}) $}

被4整除:\\
The last two digits must be divisible by 4. Tips\footnote{$A=\overline{a_{n}a_{n-1}...a_{3}} \times 100 + \overline{a_{2}a_{1}}$},后者是关键

被5整除:

被6整除:

被7整除:\\


被8整除:\\
The last three digits must be divisible by 8

被9整除:

被11整除:\\
第一位数$-$第二位数$+$第三位数$-$...,最终值如果能被11整除,正负交替,从前往后和从后往前一样,负数也可以判定. Tips\footnote{$10=11\times 1 - 1, 100=11\times 9 + 1, 1000=11\times 91-1$}

被13整除:

被17整除:

\newpage

\subsection{proof} \label{sec:proof}
被7整除\\
\textbf{方法1}:三位截断:if $A=\overline{b_{1}b_{2}b_{3}a_{1}a_{2}a_{3}}$ 能被7整除,then: $\overline{a_{1}a_{2}a_{3}} - \overline{b_{1}b_{2}b_{3}}$可以被7整除\\
$A/7 = (\overline{b_{1}b_{2}b_{3}}\times 10^{3} + \overline{a_{1}a_{2}a_{3}})/7
= 143\times \overline{b_{1}b_{2}b_{3}} + (\overline{a_{1}a_{2}a_{3}} - \overline{b_{1}b_{2}b_{3}})/7
$

\textbf{方法2}:降位,如五位数变为四位,再变为三位,再两位。\\
Suppose we have a number $A=\overline{a_{1}a_{2}a_{3}a_{4}b}$, let $A=\overline{ab}$, while $a=\overline{a_{1}a_{2}a_{3}a_{4}}$.\\
if 能化简为判断 $a+mb$能否被7整除,则简化成功:\\
\[
\begin{cases}
  a + mb = 7n_{1}\\
  10a + b = 7n_{2}
\end{cases}
\Rightarrow
(10m-1)b = 7(n_{1}-n_{2})
\]
\\
That means $10m-1$必须是7的倍数,此时$m$可以为$5$ or $-2$。那么我们就可以简化为判断$a-2b$ or $a+5b$, $-2$ 和 $5$ 正好相差7.\\
e.g.\\
$329 \to 32-2\times 9=14 $ \\
$4564 \to 456-2\times 4=448 \to 44+8\times 5=7\times 12$

推广:\\
被$11$整除:$(10m-1)|11$, $m=-1$。当然,被$11$整除还有一个更方便的方法,那就是依次加减\\
被$13$整除: $(10m-1)|13$, $m=4,-7$\\
被$17$整除:$(10m-1)|17$, $m=-5$\\
我们可以看到,m是一个呈周期循环,太大了就没有意义了,如果一个三位数,最后化简还是三位数,就没有了意义

这种方法用计算机编程来判断很方便


\newpage
\subfile{algebra-misc}

\end{document}
%%% Local Variables:
%%% mode: LaTeX
%%% TeX-master: "../main"
%%% End:
