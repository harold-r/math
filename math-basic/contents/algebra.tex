\documentclass[../main.tex]{subfiles}
\graphicspath{{\subfix{../images/}}}

\begin{document}

\part{basics}\

\section{反证法(proof by contradiction)}
英国数学家高德菲·哈罗德·哈代在他的文章《一个数学家的辩白》描述:“欧几里得最喜欢用的反证法,是数学家最精良的武器。它比起棋手所用的任何战术还要好:棋手可能需要牺牲一只兵甚至更多,但数学家却是牺牲整个棋局来获得胜利。”

反证法常用于"正面证明不容易或不能得出结果"的情况

Procedure:\\
1. The proposition to be proved is $P$\\
2. We assume $P$ to be false, i.e., we assume $\neg P$\\
3. It is shown that $\neg P$ implies falsehood. This is typically accomplished by deriving two mutually contradictory assertions. $Q$ and $\neg Q$ and appealing to the law of noncontradiction\\
4. Since assuming $p$ to be false leads to a contradiction. It's concluded that $p$ is in fact true

Example: $\sqrt{2}$是无理数的证明

假设$\sqrt{2}$是有理数,那么就可以写为$\frac{p}{q}$,其中$p, q$为正整数且互质,那么有: $p=\sqrt{2}\times q$, then $p^2=2\times q^2$, 很显然$p^2$是偶数,而只有偶数的平方才是偶数,所以$p$是偶数。假设$p=2s$,then $p^2=4s^2=2q^2 \Rightarrow q^2=2s^2$, 从而$q$也是偶数,这与互质矛盾,假设不成立,从而得证。

\section{素数、合数}

素数 vs 质数\\
直到清末,prime number一直被翻译为素数,素数、非素数、合成数都是日译名。汉语中的“素”有“根本”之义,有可能“素”与“数”读音接近,易混淆,用了“质”。但华罗庚的《堆垒素数》和陈景润研究的“哥德巴赫猜想”也是用的素数。现在的一些中小学教材,统一使用的是质数,但又标明了“质数,又叫素数”

为什么规定1不是素数?\\
$1$既不是素数也不是合数。加入$1$是素数,那么一个数比如$1\times 2^2 \times 3^3$也可以写为$1^2\times 2^2 \times 3^3$,这样分解就不唯一了

素数:大于1的自然数,如果只有1与自身两个因数,那么这个数就称为素数。如$2, 3, 5, 7, 11$ etc。2是最小的素数,也是素数中唯一的偶数

合数:大于1的自然数,如果除了1与自身以外,还有其他因数,则称此数为合数。如$4, 6, 8$ etc.

根据定义,$1$既不是素数,也不是合数。全体自然数分为:$1$、 素数、合数。

互质(互素, coprime):两个或两个以上的整数的最大公约数是$1$\\
如果数域是正整数,那么$1$与所有正整数互质\\
如果数域是整数,那么$1$ and $-1$与所有整数互质,而且他们是仅有的与$0$互质的整数\\
两个整数($a,b$)互质,记为:$a \perp b$

任何有理数都可以表示为连个互质的整数之比,也就是一个最简分数。根据有理数定义,有理数是指可以表示成两个整数之比的数,而分数的可以通过不断约分最终得到一个分子和分母互质的最简分数

专门研究数学的人认为素数是最基本的数,因为任何大于1的整数要么是素数,要么是若干素数的积。德国的高斯曾经说过:“数学是科学的皇后,数论是数学的皇冠”。费马曾说过:“全部的数论问题就在于以何种方法来讲一个整数分解质因数”。

素数是有限的还是无限的?这被欧几里得证明了,有了欧几里得定理(Euclid's theorem),是数论中的基本定理。

欧几里得定理(Euclid's theorem):\\
《几何原本》第九卷中,有以下陈述:存在着比指定的任意多个素数更多的素数。也即:素数的个数是无限的。

素数是无限的.

Proof1(欧几里得,不是反证法?):\\
1. 假设素数是有限的,那么可以假设素数只有一个有限的集合S,as $\{p_1, p_2, ... p_n\}$\\
2. 构造一个新的数: $Q=p_1\times p_2\times \cdots p_n+1$\\
3. 分析$Q$\\
\forceindent a, $Q$比$S$中的任意一个素数要大,它不在$S$内\\
\forceindent b, 用集合$S$中的任何一个素数$p_i$去除$Q$, 都会余1\\
4. 得出矛盾\\
\forceindent a, 意味着$Q$要么本身就是一个新的素数,它不在我们构造的集合$S$内\\
\forceindent b, 要么$Q$是一个合数,它可以被一个比$P_n$(我们假设的最大的素数)还要大的素数整除,根据\textbf{算数基本定理},这意味着存在一个不在$S$内的素数,这个素数比\\
5. 这与我们最初假设的“素数是有限的”矛盾,因此素数一定有无限多个

Proof2(欧几里得):\\
考虑正整数$n$的阶乘$n!$可以被$2$到$n$的所有的整数整除,$n!+1$并不能被$2$到$n$的任何自然数所整除,因此$n!+1$有两种可能性:是素数,或者能被大于$n$的素数(素数基本定理)整除,在任何一个case中,都表明至少存在一个比$n$大的素数

\textbf{素数定理}, 又称作质数定理, prime number theorem,是素数分布理论的中心定理,是关于素数个数问题的一个命题:\\


\section{排列组合(permutation and combination)}

\subsection{排列(permutation)}
Permutaion(排列、变换、置换,比如古典密码里的置换) or Arrangement,所以数学符合$P$和$A$都可以。

利用乘法原理:$A_{n}^{k}=\overbrace{n(n-1)(n-2)...(n-k+1)}^{\text{k factors}}=\frac{n!}{(n-k)!}$\\
Also use:  $P^{n}_{k}$, $P(n,k)$, $_{n}P_{k}$, $^{n}P_{k}$, $P_{n,k}$. Note the slight difference: $P^{n}_{k}$ and $A_{n}^{k}$

重复排列:从$n$个元素中取出$k$个元素,$k$个元素可以重复:$U^{n}_{k}=n^{k}$

\subsection{组合(combination)}
Combination just likes permutation, but the order doesn't matter

This formula can be derived from the fact that each k-combination of a set $S$ of $n$ members has permutations so\\
$P_{n}^{k}=C_{n}^{k}\times k!$ or $C_{n}^{k}=P_{n}^{k}/k!$. The $C_{n}^{k}$ often denoted by $\binom{n}{k}$

\section{绝对值}
\subsection{绝对值的意义}
本质是表示距离,比如数轴上的线段距离,差值的绝对值

第一要务一般是如果去绝对值,从代数上看就是要去讨论

不但要会去绝对值,还要会用绝对值列出题目相应的等式或者不等式,然后去解

又比如$|3-2x| + |x-3|$的最小值,要善于变换,以方便几何上的直观

\subsection{最值}
$f(x)= |x+1|+ |2-x|$的最值\\
$f(x) = |x+1| + |2x-1|$的最值,Tips\footnote{$|2x-1| = 2|x-\frac{1}{2}|$}\\
$f(x)= |x+1| + |x| + |x-2|$的最值\\
$f(x)=|2x-1|+|4x-3|$\\
$f(x)=||x-1|-3|$

\subsection{推广}
$f(x) = |x-a_1| + |x-a_2|+ |x-a_3| + \cdots + |x-a_{n}| $
$f(x)=|x+1|+|2x-1|$可以化简为:$f(x)=|x+1|+2|x-\frac{1}{2}|=|x+1|+|x-\frac{1}{2}|+|x-\frac{1}{2}|$

奇点偶段,证明方法:从1到3到5,从2到4到6,以至无穷

\section{整除规则(divisibility rule)}
Let $A$ = $\overline{a_{n}a_{n-1}...a_2a_1} = a_{n}\times 10^{n-1}+a_{n-1}\times 10^{n-2}...+a_2\times 10 + a_1$
\subsection{基本判别(rules)}

被2整除:\\
The last digit is even

被3整除和被9整除:\\
The sum of digits must be divisible by $3$ or $9$.
Tips\footnote{$A = a_{n}\times (9+1)^{n-1}+a_{n-1}\times (9+1)^{n-2}...+a_2\times (9+1)+a_1 \\
~~~   = (a_{n}\times 9^{n-1} + a_{n-1}\times 9^{n-2}+...+a_2\times 9)+(a_{n}+a_{n-1}+...+a_2+a_1) $}

被4整除:\\
The last two digits must be divisible by 4. Tips\footnote{$A=\overline{a_{n}a_{n-1}...a_3} \times 100 + \overline{a_2 a_1}$},后者是关键


被5整除:

被6整除:

被7整除:\\

被8整除:\\
The last three digits must be divisible by 8

被9整除:

被11整除:\\
第一位数$-$第二位数$+$第三位数$-$...,最终值如果能被11整除,正负交替,从前往后和从后往前一样,负数也可以判定. Tips\footnote{$10=11\times 1 - 1, 100=11\times 9 + 1, 1000=11\times 91-1$}

被13整除:

被17整除:

\newpage

\subsection{proofs} \label{sec:proof}
被7整除\\
\textbf{方法1}:三位截断:if $A=\overline{b_1b_2b_3a_1a_2a_3}$ 能被7整除,then: $\overline{a_1a_2a_3} - \overline{b_1b_2b_3}$可以被7整除\\
$A/7 = (\overline{b_1b_2b_3}\times 10^3 + \overline{a_1a_2a_3})/7
= 143\times \overline{b_1b_2b_3} + (\overline{a_1a_2a_3} - \overline{b_1b_2b_3})/7
$

\textbf{方法2}:降位,如五位数变为四位,再变为三位,再两位。\\
Suppose we have a number $A=\overline{a_1a_2a_3a_4b}$, let $A=\overline{ab}$, while $a=\overline{a_1a_2a_3a_4}$.\\
if 能化简为判断 $a+mb$能否被7整除,则简化成功:\\
\[
\begin{cases}
  a + mb = 7n_1\\
  10a + b = 7n_2
\end{cases}
\Rightarrow
(10m-1)b = 7(n_1-n_2)
\]
\\
That means $10m-1$必须是7的倍数,此时$m$可以为$5$ or $-2$。那么我们就可以简化为判断$a-2b$ or $a+5b$, $-2$ 和 $5$ 正好相差7.\\
e.g.\\
$329 \to 32-2\times 9=14 $ \\
$4564 \to 456-2\times 4=448 \to 44+8\times 5=7\times 12$

推广:\\
被$11$整除:$(10m-1)|11$, $m=-1$。当然,被$11$整除还有一个更方便的方法,那就是依次加减\\
被$13$整除: $(10m-1)|13$, $m=4,-7$\\
被$17$整除:$(10m-1)|17$, $m=-5$\\
我们可以看到,m是一个呈周期循环,太大了就没有意义了,如果一个三位数,最后化简还是三位数,就没有了意义

这种方法用计算机编程来判断很方便

\newpage
\subfile{algebra-misc}

\newpage
\section{todo}
生日问题
https://zh.wikipedia.org/wiki/%E7%94%9F%E6%97%A5%E5%95%8F%E9%A1%8C


\end{document}
%%% Local Variables:
%%% mode: LaTeX
%%% TeX-master: "../main"
%%% End:
