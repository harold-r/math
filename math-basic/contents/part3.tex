\documentclass[../main.tex]{subfiles}
\graphicspath{{\subfix{../images/}}}

\begin{document}
\part{Part3}
\section{整除规则(divisibility rule)}
Let $A$ = $\overline{a_{n}a_{n-1}\ldots a_2a_1} = a_{n}\times 10^{n-1}+a_{n-1}\times 10^{n-2}\ldots +a_2\times 10 + a_1$
\subsection{基本判别(rules)}

被2整除:The last digit is even

被3整除和被9整除: The sum of digits must be divisible by $3$ or $9$.
Tips\footnote{$A = a_{n}\times (9+1)^{n-1}+a_{n-1}\times (9+1)^{n-2}\ldots +a_2\times (9+1)+a_1 \\
~~~   = (a_{n}\times 9^{n-1} + a_{n-1}\times 9^{n-2}+ldots +a_2\times 9)+(a_{n}+a_{n-1}+ldots +a_2+a_1) $}

被4整除:The last two digits must be divisible by 4. Tips\footnote{$A=\overline{a_{n}a_{n-1}ldots a_3} \times 100 + \overline{a_2 a_1}$}

被8整除:The last three digits must be divisible by 8

被11整除:\\
第一位数$-$第二位数$+$第三位数$-$...,最终值如果能被11整除,正负交替,从前往后和从后往前一样,负数也可以判定. Tips\footnote{$10=11\times 1 - 1, 100=11\times 9 + 1, 1000=11\times 91-1$}

\subsection{proofs} \label{sec:proof}
被7整除:\\
\textbf{方法1}:三位截断:if $A=\overline{b_1b_2b_3a_1a_2a_3}$ 能被7整除,then: $\overline{a_1a_2a_3} - \overline{b_1b_2b_3}$可以被7整除\\
$A/7 = (\overline{b_1b_2b_3}\times 10^3 + \overline{a_1a_2a_3})/7
= 143\times \overline{b_1b_2b_3} + (\overline{a_1a_2a_3} - \overline{b_1b_2b_3})/7
$

\textbf{方法2}:降位,如五位数变为四位,再变为三位,再两位。\\
Suppose we have a number $A=\overline{a_1a_2a_3a_4b}$, let $A=\overline{ab}$, while $a=\overline{a_1a_2a_3a_4}$.\\
if 能化简为判断 $a+mb$能否被7整除,则简化成功:\\
\[
\begin{cases}
  a + mb = 7n_1\\
  10a + b = 7n_2
\end{cases}
\Rightarrow
(10m-1)b = 7(n_1-n_2)
\]
\\
That means $10m-1$必须是7的倍数,此时$m$可以为$5$ or $-2$。那么我们就可以简化为判断$a-2b$ or $a+5b$, $-2$ 和 $5$ 正好相差7.\\
e.g.\\
$329 \to 32-2\times 9=14 $ \\
$4564 \to 456-2\times 4=448 \to 44+8\times 5=7\times 12$

推广:\\
被$11$整除:$(10m-1)|11$, $m=-1$。当然,被$11$整除还有一个更方便的方法,那就是依次加减\\
被$13$整除: $(10m-1)|13$, $m=4,-7$\\
被$17$整除:$(10m-1)|17$, $m=-5$\\
我们可以看到,m是一个呈周期循环,太大了就没有意义了,如果一个三位数,最后化简还是三位数,就没有了意义

这种方法用计算机编程来判断很方便

\end{document}

%%% Local Variables:
%%% mode: LaTeX
%%% TeX-master: "../main"
%%% End:
