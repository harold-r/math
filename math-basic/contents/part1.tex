% Theorem, lemma etc.

\documentclass[../main.tex]{subfiles}
\graphicspath{{\subfix{../images/}}}

\begin{document}

\part{MISC}
数学基本思想:\\
抽象能力:会在错综复杂的事物中把握本质\\
推理能力:会在杂乱无章的事物中理清头绪\\
建模能力:会在千头万绪的事物中发现规律

数学四大基本思想:\\
函数与方程、数形结合、分类讨论和转化与化归(复杂或未知的问题,通过一定的变换,最终归结为已知或更容易解决的问题的思维方法)

数学基本方法:\\
数形结合、分类讨论、换元、数学归纳法、反证法、类比

数域(域:field):指一个数集,它对加法、减法、乘法和除法(除数不为零)运算是封闭的,并且包含$0$和$1$\\
换句话说,对数域中任意两个数进行这四种基本运算,其结果仍然属于这个数域。\\
常见的数域包括有理数域(Q)、实数域(R) 和复数域(C)\\
一个集合成为数域,需满足以下条件:\\
1. 包含0和1:数域中必须有加法单位元0和乘法单位元1\\
2. 封闭性:加法和减法封闭;乘法封闭;除法封闭\\
非数域例子:自然数集和整数集,不构成数域,因为除法运算不封闭,例如$2/3$不属于自然数或者整数

丢番图方程,又称为不定方程(Diophantine equation), Diophantus is a Greek mathematician\\
丢番图的研究在数论中占有重要地位,如丢番图方程、丢番图集合、丢番图逼近等都是数学的重要领域

最大公约数:GCD(Greatest Common Divisor) or HCF:(Highest Common Factor).\\
e.g. $gcd(3,9)=3$ , $gcd(-3,9)=3$

$0!$规定为$1$

\part{定理Theorem}

\section{算术基本定理(Fundamental theorem of arithmetic)}
算术基本定理, also called unique factorization theorem(正整数唯一分解定理) and prime factorization theorem,即:每个大于1的自然数,要么本身就是素数,要么可以写为2个或以上的素数的积,而且这些素因子按大小排列之后,写法仅有一种方式。例如: $ 1200 = 2^4\cdot 3 \cdot 5^2$

算术基本定理是初等数论中一个基本的定理,也是许多其他定理的逻辑支撑点和出发点。\\
由两部分组成:\\
1.分解的存在性\\
2.分解的唯一性,即若不考虑排列的顺序,正整数分解为素数乘积的方式是唯一的

这个定理也是为什么$1$不是质数的主要原因, 如果$1$是prime,$2=2\cdot 1=2\cdot 1\cdot 1=...$

STATEMENT:\\
Every positive integer $n>1$ can be represented in exactly one way as a product of prime powers:\\
$ n = p_1^{n_1}p_2^{n_2}\cdots p_k^{n_k} = \prod_{i=1}^kp_i^{n_i}$, where $p_1<p_2<p_3...<p_k$ are primes,and $n_i$ are positive integers.

Since $a^0=1$, any positive integer can be uniquely represented as an infinite product taken over all the positive prime numbers, as
\[
n = 2^{n1}3^{n2}5^{n3}7^{n4}\cdots = \prod_{i=1}^{\infty}P_i^{n_i}
\]

等价命题:If a prime divides the product of two integers, then it must divide at least one of these integers.That is:\\
If Prime $P|ab$, then, either $P|a$ or $P|b$

Proof:

1. Existence\\
用反证法:假设存在大于$1$的自然数不能写为素数的乘积,把最小的那个称为$n$\\
n不可为素数,因为$n=n$,可以写成素数的成绩,因此$n$一定是合数,而每个合数都可以分解为两个严格小于自身而大于$1$的自然数的乘积。设$n=a\times b$,根据假设,$n$是最小的不能写为素数乘积的自然数,$a<n, b<n$,所以$a=p_1p_2\times p_n, b=q_1q_2\cdots q_n$, and $n=ab=p_1p_2\cdots p_nq_1q_2\cdots q_n$可以写为素数的乘积,由此产生矛盾,故大于$1$的自然数必可以写为素数的乘积

2. Uniqueness\\
欧几里得引理:if $p|ab$, either $p|a$ or $p|b$


\todo{todo...}

\section{代数基本定理(Fundamental theorem of algebra)}
Also called "d'Alembert–Gauss theorem"\\
描述为:任何一个复系数的一元$n$ 次多项式方程($n\geq 1$),至少有一个复数根。

有时候这个定理描述为:任何一个非零的一元n次复系数多项式,都正好有$n$个复数根(重根视为多个根)。但实际上,是“至少有一个根的”直接结果,因为把多项式除以它的线性因子可以推出。也就是说,任何一个$n$次多项式,都可以因式分解为$n$个复系数一次多项式的乘积(根据多项式除法\todo{Proof?})。

推论:任何一个非零的一元n次复系数多项式,都正好有n个复数根(重根视为多个根)。

意义:复数域是代数封闭的;该定理是代数学和近世代数中的一个基础性结论

尽管这个定理被命名为“代数基本定理”,但它还没有纯粹的代数证明,许多数学家都相信这种证明不存在。另外,它也不是最基本的代数定理;因为在那个时候,代数基本上就是关于解实系数或复系数多项式方程,所以才被命名为代数基本定理。

所有的证明都包含了一些数学分析,至少是实数或复数函数的连续性概念。有些证明也用到了可微函数,甚至是解析函数。


\section{多项式除法、多项式余式定理(Polynomial division, Polynomial remainder theorem)}

\begin{flalign*}
  \frac{P(x)}{D(x)} = Q(x)+\frac{R(x)}{D(x)} \Rightarrow P(x)=D(x)Q(x)+R(x)&&
\end{flalign*}

If $D(x)=x-a$, then $P(x)=(x-a)Q(x)+R(x) = (x-a)Q(x)+r$\\
根据定义,$R(x)$的次数小于$1$, so $R(x)$只能为常数\\
$\Rightarrow P(a)=(a-a)Q(x)+r=r$\\
得到\textbf{多项式余式定理}:多项式$P(x)$除以$x-a$所得的余式$=P(a)$

dividend = divisor x quotient + reminder

Examples:\\
Let $f(x)=x^3-12x^2-42$, divided by $x-3$, gives the quotient $x^2-9x-27$, and the remainder $-123$.\\
By the polynomial remainder theorem, $f(3)=-123$

寻找多项式的切线? \todo{?直觉要用微积分,但是这个是啥情况?} \url{https://zh.wikipedia.org/wiki/%E5%A4%9A%E9%A1%B9%E5%BC%8F%E9%99%A4%E6%B3%95}

\section{因式定理(Factor theorem)}
The Factor theorem connects polynomial factors with polynomial roots.(关于多项式的因式和零点的定理)

一个多项式$f(x)$有一个因式$ax-b$当且仅当 $f(\frac{b}{a})=0$

普遍应用于因式分解,利用长除法,除以零点$(x-a)$

Example:\\
分解因式:$(x-y)^3+(y-z)^3+(z-x)^3$\\
$x=y, y=x, x=z$是$0$点,so $k(x-y)(y-z)(x-z)$, let $x=0,y=1,z=2 \Rightarrow k=3$

\section{因式分解定理}
数域$F$上的每个次数$\ge 1$的多项式$f(x)$都可以分解为数域$F$上一些不可约多项式的乘积,并且是唯一的,即:\\
$f(x)=p_{x}(x)p_2(x)p_3(x)\cdots p_{s}(x) = q_1(x)q_2(x)q_3(x)\cdots q_{t}(x)$,其中$p_{i}(x)$和$q_{j}(x)$都是数域$F$上的不可约多项式,那么必有$s=t$,而且可以适当排列因式的次序,使得\\
$p_{i}(x)=c_{i}q_{i}(x)$

分解方法: 公因式、公式法、分组分解、拆添项、十字交叉、一次因式检验法(有理根定理)

\section{有理根定理(Rational root theorem)}
Also called rational root test, rational zero theorem, rational zero test or $p/q$ theorem

描述:对于$a_{n}x^{n} + a_{n-1}x^{n-1}+\cdots + a_1x+a_{0}=0$, \colorbox{BurntOrange}{系数$a_{i}\in \mathbb{Z}$}, and $a_{0}, a_{n} \neq 0$.\\
该定理指出,如果存在有理根$x=\frac{p}{q}$, written in lowest term(that is $p$ and $q$ are relatively prime, 互质),满足:\\
$p$是$a_{0}$的整数因子,i.e. $p|a_{0}$.整除符号,Tips\footnote{$a|b$: $a$整除$b$,$b$能被$a$整除,也就是$b$除以非零$a$,商是一个整数. i.e. $2|6$}\\
$q$是$a_{n}$的整数因子,i.e. $q|a_{n}$.

该定理是高斯定理关于多项式分解的一个特例

Proof:\\
Let $P(x) = a_nx^n+a_{n-1}x^{n-1}+\cdots +a_1x+a_0$ with $a_i \in \mathbb{Z}$, $a_0,a_n \neq 0$

Suppose $P(p/q) = 0$ for some coprime $p,q \in \mathbb{Z}$:\\
$P(\frac{p}{q}) = a_{n}(\frac{p}{q})^{n} + a_{n-1}(\frac{p}{q})^{n-1}+\cdots + a_1(\frac{p}{q}) + a_{0} = 0$\\
$\Rightarrow a_{n}p^{n}+ a_{n-1}p^{n-1}q+\cdots +a_1pq^{n-1}+ a_{0}q^{n} = 0$\\
$\Rightarrow p(a_{n}p^{n-1}+a_{n-1}qp^{n-2}+\cdots +a_1q^{n-1}) = -a_{0}q^{n} \Rightarrow ()=-a_{0}\frac{q^{n}}{p}$ \\
$\Rightarrow q(a_{n-1}p^{n-1}+a_{n-2}qp^{n-2}+\cdots +a_{0}q^{n-1}) = -a_{n}p^{n} \Rightarrow ()=-a_{n}\frac{p^{n}}{q}$\\
我们注意到,括弧内是整数,因为$a_{i}$是整数,所以这是关键

$p,q$互质,$\frac{p}{q} = \pm \frac{a_0\text{的因子}}{a_n\text{的因子}}$

注意$p, q$为$1$的特殊情况,显而易见$1$永远是第一个选择

关键点:\\
1. 系数是整数\\
2. 如果存在有理根,则必符合此定理,否则存在无理根(如$\sqrt{89}$)亦或者复数根

Examples:\\
$x^3-7x+6=0$: \\
有理根有可能是:$\pm \frac{\{1,2,3,6\}}{1}=\pm {1, 2, 3, 6}$,恰好$1, 2, -3$,所以也恰好可以写为:$(x-1)(x-2)(x+3)=0$

$3x^3-5x^2+5x-2=0$, 如果有有理根,则必在$\pm \frac{1,2}{1,3}=\pm{1,2, \frac{1}{3}, \frac{2}{3}}$中。8个候选根,需要测试8次,最后才知道$x=2/3$是唯一有理根.\\
很是繁琐不是?所以可以通过评估$P(r)$来测试缩小范围(比如使用秦九韶算法?)。\\
Firstly, if $x<0$, the $P$ will be negative, so every root is positive\\
$P(1)=1$, so 1 is not the root. Moreover, if one sets $x=1+t$, so $Q(t)=P(1+t)$, 展开后,三次项是3,一次项是1,implies $Q$ must belongs to ${\pm 1, \pm \frac{1}{3}}$, and $P$ satisfy $x=1+t \in {2,0,4/3,2/3}$. 再次显示必须为正,两个候选项是$2, 2/3$, 将$2$带入,显然不是,最后测试$2/3$\

If $a, b$ and $\frac{a^2}{b}+\frac{b^2}{a}$ are integers, then both $\frac{a^2}{b}$ and $\frac{b^2}{a}$ must be integers.\\

% see https://en.wikipedia.org/wiki/Rational_root_theorem

\section{韦达定理(Vieta's formulas)}
Any general polynomial of degree $n, P(x)=a_{n}x^{n}+ a_{n-1}x^{n-1}+\cdots +a_1x+a_{0}$, by the "fundamental theorem of algebra", roots are $x_1, x_2, x_3\cdots$
\[
  \left\{
    \begin{array}{l}
      x_1+x_2+x_3+\cdots + x_{n-1}+ x_{n} = -\frac{a_{n-1}}{a_{n}}\\
      (x_1x_2+x_1x_3+x_1x_4+\cdots +x_1x_{n})+(x_2x_3+x_2x_4+\cdots +x_2x_{n})+\cdots + x_{n-1}x_{n} = \frac{a_{n-2}}{a_{n}}\\
      \vdots\\
      x_1x_2x_3\cdots x_{n} = (-1)^{n}\frac{a_{0}}{a_{n}}
    \end{array}
  \right.
\]

\subsection{Proof}
$a_{n}x^{n}+a_{n-1}x^{n-1}+\cdots + a_1x + a_{0} = a_{n}(x-x_1)(x-x_2)\cdots (x-x_{n}) $\\
展开后比较系数\\
\[
  \left\{
    \begin{aligned}
      a_{n-1} &= -a_{n}(x_1+x_2+\cdots + x_{n-1}+ x_{n})\\
      a_{n-2} &= a_{n}[(x_1x_2+x_1x_3+\cdots +x_1x_{n})+ (x_2x_3+ x_2x_4+\cdots + x_2x_{n})+\cdots + x_{n-1}x_{n}]\\
      \vdots \\
      a_{0} &= (-1)^{n}a_{n}x_1x_2\cdots x_{n}
    \end{aligned}
  \right.
\]

\subsection{韦达定理的逆定理}
对于一元二次方程

利用圆来研究一元二次方程?  http://202.175.82.54/tplan/2006/intro/R027.pdf


\subsection{Examples}
If $n=2$(quadratic), $ax^2+bx+c=0 = a(x-x_1)(x-x_2)$展开比较即有,也可以用求根公式

if $n=3$, $x_1, x_2, x_3$是$ax^3+bx^2+cx+d=0$的三个根, then:\\
$ax^3+bx^2+cx+d = a(x-x_1)(x-x_2)(x-x_3)=a(x^3-(x_1+x_2+x_3)x^2+(x_1x_2+x_2x_3+x_1x_3)x- x_1x_2x_3=0)$, That is\\
$x_1+x_2+x_3= -\frac{b}{a}, x_1x_2+x_1x_3+x_2x_3=\frac{c}{a}, x_1x_2x_3=-\frac{d}{a}$

\section{二项式定理(Binomial theorem)}
\[
  (x+y)^{n}=C_{n}^{0}x^{n}y^{0}+C_{n}^1x^{n-1}y^1+C_{n}^2x^{n-2}y2+\cdots +C_{n}^{n}x^{0}y^{n}
\]

Examples:

$(x+y)^{0}=1$\\
$(x+y)^1=x+y$\\
$(x+y)^2=x^2+2xy+y^2$\\
$(x+y)^3=x^3+3x^2y+3xy^2+y^3$\\
$(x+y)^4=x^4+4x^3y+6x^2y^2+4xy^3+y^4$

$(x+y)^3=xxx+xxy+xyx+xyy+yxx+yxy+yyx+yyy$, total $2^3$ terms

Let $x=y=1$, we have $2^{n}=C_{n}^{0}+C_{n}^1+\cdots +C_{n}^{n}$

Proof:\\
Method 1:数学归纳法(inductive proof)

Method 2: 组合方法

$(a+b)^{n}=\overbrace{(a+b)(a+b)\cdots (a+b)}^{\text{n terms}}$, $n$个括号相乘,从$n$个选出$k$个括号中的$a$,再从剩余的$n-k$个括号中选出$(n-k)$个$b$,得到一组$a^{k}b^{n-k}$,而这种选法共有$C_{n}^{k}$种,故总共有$C_{n}^{k}$个$a^{k}b^{n-k}$;其他同理

More:\\
$(1+x)^{-1}$

$(1+x)^{\frac{1}{2}}$

$(1+\frac{1}{n})^{n}$

Multinomial theorem:\\
$(x_1+x_2\cdots +x_{m})^{n}$

\section{鸽巢原理(Pigeonhole principle)}

鸽笼原理,又名狄利克雷抽屉原理、鸽巢原理。

表述1:若有$n$个笼子和$n+1$只鸽子,所有鸽子都被放在鸽笼里,那么至少有一只笼子有至少$2$只鸽子

表述2:若有$n$个笼子和$kn+1$只鸽子,所有鸽子都被放在鸽笼里,那么至少有一只笼子有至少$k+1$只鸽子

集合论的表述:若$A$是$n+1$个原色,$B$是$n$元集,则不存在从$A$到$B$的单射

推广:如果把$n$个对象分配到$m$个容器中,必有一个容器容纳至少$\frac{n}{m}$个对象

反证法证明此原理

例子:

北京至少有两个人头发数是一样多。常人头发大概是$15$万左右,假定没有人的头发超过$100$万,北京人口大于$100$万。

有$n$个人(至少两人)互相握手(随意找人握),必有两人握过手的人数相同

这个原理经常在计算机中得到真正的应用,比如哈希表的重复问题是不可避免的,因为keys的数目总是比indices的数目多,什么算法都不可能解决

这个原理,还证明任何无损压缩算法,在把一些输入变小的同时,作为代价一定有其他的输入增大,否则对于长度为L的输入集合,该压缩算法总能将其映射到一个更小的长度小于L的输出集合,而这与鸽巢理论相悖

\todo{??....}

\section{裴蜀定理(Bézout's identity )}

\begin{mdframed}
\textbf{Bézout's identity(Bézout's lemma)}: Let $a$ and $b$ be integers with greatest common divisor $d$, Then there exist integers $x$ and $y$ such that $ax+by=d$. Moreover, the integers of the form $az+bt$ are exactly the multiples of $d$
\end{mdframed}


\end{document}
%%% Local Variables:
%%% mode: LaTeX
%%% TeX-master: "../main"
%%% End:
