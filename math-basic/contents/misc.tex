\documentclass[algebra.tex]{subfiles}
\graphicspath{{\subfix{../images/}}}

\begin{document}
\part{MISC}
\section{misc}

坐标的平移\\
在高考、中考、合格考等考试的手册中,有这些时间点需要规范化语音,比如:提前35分钟入场提示,提前20分钟宣读考场规则,提前15分钟启封试卷袋,提前5分钟发试卷,提前0分钟也就是到点语音“开始答题”,那么我们可以用数轴上的点来表示,$-35, -20, -15, -5, 0$,如果做音频文件,这些点就可以以$-35$为原点,也就是数轴的原点从原来的$0$平移到$-35$,那么我们得到的关键点为:$0,15,20,30,35$。往左移就是加,往右移就是减。

对于圆锥曲线一般方程:$Ax^2+Bxy+Cy^2+Dx+Ey+F=0$(里面的$xy$项是旋转项),我们总是可以通过平移、旋转,去掉$xy, x, y$项。


直角梯形,对于勾股定理、三角函数的和差

\section{因式分解}
KEY:十字交叉、拆添项、有理根定理、欧几里得算法、单位根因式、待定系数法(通常用于四次五次)、换元法

对于二次多项式,首先要考虑的是十字交叉,然后拆添项。二次以上的,可以用以下的一些技巧。

\textbf{待定系数法:}例 $x^4+9x^3+16x^2-x-3$\\
如上例:有理根定理尝试后发现都不行,也就是应该无有理根,那也意味着没有一次的多项式因式,那么可以假设可以分为两个二次的多项式$(x^2+ax-1)(x^2+bx+3)$,最后列方程即可。\\
对于四次五次型的感觉挺合适的,包括上面的一些四次型。

\textbf{单位根:}满足$z^n=1$的所有复数$z$叫做$n$次单位根\\
最简单的一种形式是$x^3=1$, $(x-1)(x^2+x+1)=0$,$x^2+x+1$也可叫做单位根因式,很明显if $\omega$是$x^2+x+1$的根, then $\omega^3=1$\\
其实$x^3=-1 \Rightarrow (x+1)(x^2-x+1)=0$中的$x^2-x+1$也是可以用的,比如$x^4+x^3+x^2+2$,将$x^3=-1$带入正好得$x^2-x+1$。

$6x^3-11x^2+x+4$, 根据有理根定理,先尝试$\pm 1$,易知$1$是一个根,那么$(x-1)$就是其中一个因式,又根据欧几里得算法,只要拼凑出$(x-1)$的项,那么剩余的肯定也含有$(x-1)$的项,所以我们从最高次三次着手,$(6x^3-6x^2)-(5x^2-x-4)$,这样不需要长除法,可以迅速解决。同样的$x^3+2x-3$可配成$(x^3-x^2)+(x^2+2x-3)$ or $(x^3-1)+(2x-2)$

$x^4-7x^2+1$, 根据有理根定理,尝试$\pm 1$后,发现它没有有理根,那么我们只能拼凑了, 观察看,常数项是平方项,可以从常数项着手,$x^4+2x^2+1-9x^2$。

$x^4-9x^2+64$,如果把$x^2$看作一个变量,那么明显$x^2$无解,有理根定理也不能用了。只能凑了,常数项是平方项,那么我们从常数项着手即可。

$x^4+x^3+x^2+2$,无有理根,可以从"444"着手,$(x^4+1^2x^2+1^2)+x^3+1$,也可以简单尝试$x^3=-1$正好$x^2-x+1$项。

$x^4+y^4+(x+y)^4$,往"444"去凑

$(x^2-3x+4)^2+x^3+2$,展开明显很复杂,但是可以分析一下,常数项是$18$,尝试后有理根定理,发现也不行,只能凑,平方差是能最检查运算的,凑括弧内的完全平方显得有点拙劣,尝试拼凑常数项,一个平方项的凑平方差,一个立方项凑立方和。i.e. $a^3-b^2$, Let $a=3, \; b=5$

$(x^2-1)(x+3)(x+6)+12$,常数项是$-3$,尝试后发现没有有理根,又得东拼西凑了,注意前面四个因式$(x-1)(x-3)(x-1)(x+5)=(x^2-4x+3)(x^2-4x-5)$

$x^7+x^5+1$,无有理根,既然是无理根,观察单位根是否适用,将$x^3=1$代入,那么降次后正好是$x+x^2+1=0$,那么就往单位根因式$x^2+x+1$之类的凑了(或者长除法), $x^5(x^2+x+1)-(x^6-1)$得解

$x^{10}+x^5+1$,单位根因式去检验(三次方为$1$),直接用单位根去除。

$x^2(y-z)+y^2(z-x)+z^2(x-y)$,容易看出是轮换是,当$x=y$时原式为$0$,故有因式$(x-y)$,同理有$(y-z)$, $(z-x)$,再设一个常数项得系数为$-1$。

$x^3-3x^2+3x+1-y^3$,以$x$为主变量看,存在孤零零的$1-y^3$常数项,所以$y^3$是突破口,尝试前面应该是个立方才行。

\section{最值}
\dotfill

\textbf{特征:直角坐标系下,动点与两定点之间的和或者差}

$\sqrt{x^2+4} + \sqrt{(x-8)^2+16}$\\
1.利用直角坐标系点与点之间的距离和, $(x-0)^2+(0-(-2))^2 $and $(x-8)^2+(0-4)^2$\\
2.构造直角三角形(但是往往与定义域不符,$\sqrt{x^2+4} + \sqrt{(8-x)^2+16}$)

此类也可以变形为:已知$x+y=12$,求$\sqrt{x^2+4}+\sqrt{y^2+9}$的最值

$\sqrt{x^2+4} - \sqrt{(x-8)^2+16}$,三角形两边差

但是如果$x$的系数不同呢?比如:$\sqrt{2x^2+4} \pm \sqrt{(x-8)^2+16}$\\
根据Google AI的分析结果来看,初高中范围内没有办法,需要求驻点。

\dotfill

\textbf{柯西不等式:两个方向,积或者平方和,注意探讨等号}

$x_1y_1+x_2y_2+x_3y_3 \le \sqrt{x_1^2+x_2^2+x_3^2} \sqrt{y_1^2+y_2^2+y_3^2}$ or\\
$(x_1y_1+x_2y_2+x_3y_3)^2 \le (x_1^2+x_2^2+x_3^2)(y_1^2+y_2^2+y_3^2)$

$\sqrt{x-1} + 2\sqrt{3-x}$, tips\footnote{$\sqrt{x-1} + 2\sqrt{3-x} \le \sqrt{1^2+2^2}\sqrt{x-1+3-x}$,等号需要讨论.}


$x,y,z>0$ and $x+y+z=1$,求$\frac{1}{x}+\frac{9}{y}+\frac{25}{z}$的最小值, tips\footnote{$(\sqrt{x}+\sqrt{y}^2+\sqrt{z}^2)(\frac{1}{\sqrt{x}^2}+\frac{3^2}{\sqrt{y}^2}+\frac{5^2}{\sqrt{z}^2})$}

$x,y,z >0$, $\frac{x+y+z}{\sqrt{x^2+2y^2+4z^2}}$的最大值, tips\footnote{$(x^2+\sqrt{2}^2 y^2+4z^2)(1+\frac{1}{\sqrt{2}^2} +\frac{1}{2^{2}} \ge (x+y+z))$}

求证:$\sqrt{4x+4} + \sqrt{2x-3} +\sqrt{15-3x} < \sqrt{78}$, tips\footnote{柯西后,讨论等号不成立}

$\sqrt{x}+\sqrt{27+x} +\sqrt{13-x}$, tips\footnote{使用柯西需要凑, $\sqrt{13-x} = \frac{1}{\sqrt{2}}\sqrt{26-2x}$}

已知$x^2+y^2=3$, 求$2x+3y$的最大值, tips\footnote{$(x^2+y^2)(2^2+3^2) \ge (2x+3y)^2$,但是如果需要求最小值呢?显然数形结合更合适}

\dotfill

$\frac{\sqrt{x^{4}+x^{2}+1} - \sqrt{x^{4}+1}}{x}$的max. tips\footnote{1讨论正负,2换元,3分子有理化}

\dotfill

$x,y > 0$, and $x+3y=x^{3}y^{2}$, min of $\frac{3}{x} + \frac{2}{y}$, tips\footnote{把$1/y$带入}

\dotfill


\end{document}
%%% Local Variables:
%%% mode: LaTeX
%%% TeX-master: "../main"
%%% TeX-engine: xetex
%%% End:
