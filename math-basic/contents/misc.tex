\documentclass[algebra.tex]{subfiles}
\graphicspath{{\subfix{../images/}}}

\begin{document}
\section{MISC}

坐标的平移\\
在高考、中考、合格考等考试的手册中,有这些时间点需要规范化语音,比如:提前35分钟入场提示,提前20分钟宣读考场规则,提前15分钟启封试卷袋,提前5分钟发试卷,提前0分钟也就是到点语音“开始答题”,那么我们可以用数轴上的点来表示,$-35, -20, -15, -5, 0$,如果做音频文件,这些点就可以以$-35$为原点,也就是数轴的原点从原来的$0$平移到$-35$,那么我们得到的关键点为:$0,15,20,30,35$。往左移就是加,往右移就是减。

对于圆锥曲线一般方程:$Ax^2+Bxy+Cy^2+Dx+Ey+F=0$(里面的$xy$项是旋转项),我们总是可以通过平移、旋转,去掉$xy, x, y$项。


直角梯形,对于勾股定理、三角函数的和差

\dotfill

$\frac{\sqrt{x^{4}+x^{2}+1} - \sqrt{x^{4}+1}}{x}$的max. tips\footnote{1讨论正负,2换元,3分子有理化}

\dotfill

$x,y > 0$, and $x+3y=x^{3}y^{2}$, min of $\frac{3}{x} + \frac{2}{y}$, tips\footnote{把$1/y$带入}

\dotfill


\end{document}
%%% Local Variables:
%%% mode: LaTeX
%%% TeX-master: "../main"
%%% TeX-engine: xetex
%%% End:
